%%
%% $Id: dibbler-user-faq.tex,v 1.11 2007-09-06 23:10:40 thomson Exp $
%%

\newpage
\section{Frequently Asked Questions}

Soon after initial Dibbler version was released, feedback from user
regarding various things started to appear. Some of the questions were
common enough to get into this section.

\subsection{Common Questions}

%% Q1
\Q Why client does not configure routing after assigning addresses, so
I cannot e.g. ping other hosts?

\A It's a common misunderstanding. DHCPv4 provides many configuration
parameters to host, with default router address being one of
them. Things are done differently in IPv6. Routing configuration is
supposed to be conveyed using Router Advertisements (RA) messages,
announced periodically by routers. Hosts are supposed to listen to
those messages and configure their routing appropriately. Note that
this mechanism is completely separate from DHCPv6. It may sound a bit
strange, but that's the way it was meant to work.

Properly implemented clients are supposed to configure leased address
with /128 prefix and learn the actual prefix from RA. As this is
incovenient, many clients (with dibbler included) bend the rules and
configure received addresses with /64 prefix. Please note that this
value is arbitrary chosen and may be improper in many scenarios.

\Note This behaviour has changed in the 0.5.0 release. Previous
releases configured received address with /128 prefix. To restore old,
more RFC conformant behavior, see \emph{strict-rfc-no-routing}
directive in the Section \ref{client-cfg-file}.

There was a proposed solution in a form of \cite{draft-route-option}
draft. See section \ref{feature-routing}. Unfortunately, MIF working
group in IETF decided to abandon this work.

%% Q2
\Q I would like to have the ability to reserve specific addresses for
clients with given MAC address. That's a basic and very common feature
in DHCPv4 server. Why it is not supported in Dibbler? Are there plans
to implement such feature?

\A No. It is not and will not be supported. For couple of reason. The
first and most important is that DHCPv6 identifies clients based on
DUIDs, rathar than MAC addresses. That is a protocol design choice. Of
course that does not prevent many users from saying ``I don't care, I
want MAC classification anyway!''. So here are more technical reasons
why MAC classification is a bad idea. The first technical reason is
that Dibbler couldn't be extended, because MAC address is often not
available. There are 3 possible ways a server could possibly
learn client's MAC address:
\begin{enumerate}
\item from Ethernet frame. That won't work if traffic goes through
relay
\item from DUID-LL or DUID-LLT. RFC3315 forbids looking into the DUID.
Besides of being a wrong thing to do, that also won't work, because
client with a given MAC address can use different DUID type, e.g.
DUID-EN or DUID-UUID (or others, I saw on the wire some device with DUID
type 14. Strange, uncommon, but valid).
\item using source address and extracting MAC from link-local address
thaty is based on EUI-64 that contains MAC address. That should be
available for direct traffic (simply src address of the UDP packet) or
relayed (peer addr field in RELAY-FORW message). This would usually
work, but there are cases when it won't. First, if client uses privacy
extensions (RFC4941). The other one if client and server support
unicast, some traffic will be sent from client global address, not
using link-local address at all.
\end{enumerate}

So instead of doing MAC based reservations, Dibbler supports
link-layer address based reservations. In most cases it will be
equivalent to MAC reservations. The only case where it won't work will
be with unicast, but that can be solved easily (don't use link-layer
reservations and unicast together). Despite this shortcoming,
link-layer was implemented after Dibbler 0.8.2 was
released. See \ref{feature-exceptions}
and \ref{example-server-exceptions} sections.

%% Q3
\Q Dibbler server receives SOLICIT message, prints information about
ADVERTISE/REPLY transmission, but nothing is actually transmitted. Is
this a bug?

\A Are you sure that your client is behaving properly and responds to
Neighbor Discovery (ND) requests? Before any IPv6 packet (that
includes DHCPv6 message) is transmitted, recipient reachabity is
checked (using Neighbor Discovery protocol \cite{rfc4861}). Server
sends Neighbor Solicititation message and waits for client's Neighbor
Advertisement. If that is not transmitted, even after 3 retries,
server gives up and doesn't transmit IPv6 packet (DHCPv6 reply, that
is) at all. Being not able to respond to the Neighbor Discovery
packets may indicate invalid client behavior.

%% Q4
\Q Dibbler sends some options which have values not recognized by the
Ethereal/Wireshark or by other implementations. What's wrong?

\A DHCPv6 is a relatively new protocol and additional options are in a
specification phase. It means that until standarisation process is
over, they do not have any officially assigned numbers. Once
standarization process is over (and RFC document is released), this
option gets an official number.

There's pretty good chance that different implementors may choose
diffrent values for those not-yet officialy accepted options. To
change those values in Dibbler, you have to modify file
misc/DHCPConst.h and recompile server or client. See Developer's
Guide, section \emph{Option Values} for details.

%% Q5
\Q I can't get (insert your trouble causing feature here) to
work. What's wrong?

\A Go to the project \href{http://klub.com.pl/dhcpv6/}{homepage} and
browse \href{http://klub.com.pl/lists/dibbler/}{list archives}. If
your problem was not reported before, please don't hesitate to write
to the
\href{http://klub.com.pl/cgi-bin/mailman/listinfo/dibbler}{mailing
  list}. Author prefers to not be contacted directly, but rather
over mailing list. The only exception are a security issues. In such
case, please \href{mailto:thomson(at)klub.com.pl}{contact author}
directly.

%% Q6
\Q Why is feature X not implemented? I need it!

\Q The short answer is : We do accept patches.

The longer one is more complicated. Dibbler is a hobby project with
the only developer having very limited time to dedicate. There are
many requests, bugs and missing features and I have to prioritize
them. My personal judgement about importance, difficulty, amount of
work required and and other factors of specific feature decides on
priority, compared to other features. Personal preference plays a role
here as well.

If you don't want to wait, get your hands dirty and implement it
yourself! It is not as difficult as it sounds. Dibbler code is
reasonably well documented, so understanding how it works is not that
difficult. See Dibbler Developer's Guide for introduction, code
overview, architecture etc. You can always as on dibbler-devel mailing
list.

Finally, if you are disappointed with the pace of progress (or even
lack of thereof), there are couple of things you can do. First and
foremost, consider alternatives. There is ISC DHCP implementation that
supports DHCPv6 \href{http://www.isc.org/software/dhcp}{http://www.isc.org}.
It is open source as well, but ISC provides paid support for it, if
you need one. ISC also does custom development contracts, should you
need it. ISC is a nice non-profit company, so your money will be used
for a good cause.

\subsection{Linux specific questions}

%% Q1
\Q I can't run client and server on the same host. What's wrong?

\A First of all, running client and server on the same host is just
plain meaningless, except testing purposes only. There is a problem
with sockets binding. To work around this problem, consult Developer's
Guide, Tip section how to compile Dibbler with certain options.

%% Q2
\Q After enabling unicast communication, my client fails to send
REQUEST messages. What's wrong?

\A This is a problem with certain kernels. My limited test capabilites
allowed me to conclude that there's problem with 2.4.20
kernel. Everything works fine with 2.6.0 with USAGI patches. Patched
kernels with enhanced IPv6 support can be downloaded from
\url{http://www.linux-ipv6.org/}. Please let me know if your kernel
works or not.

\subsection{Windows specific questions}

\Q Dibbler doesn't receive anything on Windows 7 or Windows 8. Is it
broken?

\A Make sure your firewall allows the traffic through. Dibbler server
must be able to receive traffic on UDP port 547. Dibbler client must
be able to receive traffic on UDP port 546. If DNS Update mechanism is
used, Dibbler must be able to send traffic to TCP and/or UDP port 53
(DNS). There are many ways in which Windows firewall can be configured
to allow such traffic. For example, in Windows 8, one can use the
following commands (assuming DNS server is located at 2001:db8:1::1"):
\begin{lstlisting}
netsh -c advfirewall
> firewall
> add rule name="dhcpv6in" dir=in action=allow localport=547 protocol=udp
> add rule name="dhcpv6out" dir=out action=allow localport=547 protocol=udp
> add rule name="ddnsout" dir=out action=allow remoteip="2001:db8:1::1"
\end{lstlisting}

%% Q1
\Q After installing \emph{Advanced Networking Pack} or \emph{Windows XP
  ServicePack2} my DHCPv6 (or other IPv6 application) stopped
working. Is Dibbler compatible with Windows XP SP2?

\A Both products (Advanced Networking Pack as well as Service Pack 2
for Windows XP) provide IPv6 firewall. It is configured by default to
reject all incoming IPv6 traffic. You have to disable this
firewall. To disable firewall on the ``Local Area Connection''
interface, issue following command in a console:

\begin{lstlisting}
netsh firewall set adapter "Local Area Connection" filter=disable
\end{lstlisting}

%% Q2
\Q Server or client refuses to create DUID. What's wrong?

\A Make sure that you have at least one up and running interface with
at least 6 bytes long MAC address. Simple ethernet or WIFI card
matches those requirements. Note that network cable must be plugged
(or in case of wifi card -- associated with access point), otherwise
interface is marked as down.

%% Q3
\Q Is Microsoft Windows 8 supported?

\A Unfortunately, Windows 8 is not supported yet. I do not have time
to run tests on Windows8, but if it provides the same API as previous
versions do, there's pretty good chance that Dibbler will work on
Windows 8.
