\newpage
\section{Relay configuration}
\label{relay-conf}
Relay configuration is stored in \verb+relay.conf+ file in the
\verb+/etc/dibbler/+ directory (Linux systems) or in current directory
(Windows systems).

\subsection{Global scope}

Every option can be declared in global scope.
Config file consists of global options and one or more inteface
definitions. Note that reasonable minimum is 2 interfaces, as defining
only one would mean to resend messages on the same interface.

\subsection{Interface declaration}

Interface can be declared this way:
\begin{lstlisting}
iface interface-name
{
  interface options
}
\end{lstlisting}

or

\begin{lstlisting}
iface number
{
  interface options
}
\end{lstlisting}

where name\_of\_the\_interface denotes name of the interface and
number denotes it's number. It does not need to be enclosed in
single or double quotes (except windows cases, when interface name
contains spaces).

\subsection{Options}

Every option has a scope it can be used in, default value and
sometimes allowed range.

\begin{description}
 \item[log-level] -- (scope: global, type: integer, default: 7) Defines
            verbose level of the log messages. The valid range if
            from 1 (Emergency) to 8 (Debug). The higher the logging
            level is set, the more messages dibbler will print.
 \item[log-name] -- (scope: global, type: string, default: Client). Defines
            name, which should be used during logging.
 \item[log-mode] -- (scope: global, type: short, full or precise,
            default value: full) Defines logging mode. In the
            default, full mode, name, date and time in the h:m:s format
            will be printed. In short mode, only minutes and
            seconds will be printed (this mode is useful on
            terminals with limited width). Recently added precise
            mode logs information with seconds and microsecond
            precision. It is a useful for finding bottlenecks in
            the DHCPv6 autoconfiguration process.
\item[interface-id-order] -- (scope: global, type: before, after or omit,
        default: before) Defines placement of the
        interface-id option. Options can be placed in the \msg{RELAY-FORW}
        message is arbitrary order. This option has been specified to control
        that order. \opt{interface-id} option can be placed before or after
        \opt{relay-message} option. There is also possibility to instruct
        server to omit the \opt{interface-id} option altogether, but since
        this violates \cite{rfc3315}, it should not be used. In general, this
        configuration parameter is only useful when dealing with buggy relays,
        which can't handle all option orders properly. Consider this parameter
        a debugging feature. Note: similar parameter is available in the dibbler-server.

\item[client multicast] -- (scope: interface, type: boolean, default: false)
        This command instructs dibbler-relay to listen on this particular interface
        for client messages sent to multicast (ff02::1:2) address.
\item[client unicast] -- (scope: interface, type: address, default: not defined)
        This command instructs dibbler-relay to listen to messages sent to a specific
        unicast address. This feature is usually used to connect multiple relays
        together.
\item[server multicast] -- (scope: interface, type: boolean, default: false)
        This command instructs dibbler-relay to send messages (received on any interface)
        to the server multicast (ff05::1:3) address. Note that this is not the same
        multicast address as the server usually listens to (ff02::1:2). Server must be
        specifically configured to be able to receive relayed messages.
\item[server unicast] -- (scope: interface, type: address, default: none)
        This command instructs dibbler-relay to send message (received on any interface)
        to speficied unicast address. Server must be properly configured to to be able to
        receive unicast traffic. See \emph{unicast} command in the \ref{example-server-unicast}
        section.
\item[interface-id] -- (scope: interface, type: integer, default: none)
        This specifies identifier of a particular interface. It is used to generate
        \opt{interface-id} option, when relaying message to the server. This option
        is then used by the server to detect, which interface the message originates from.
        It is essential to have consistent interface-id defined on the relay side and
        server side. It is worth mentioning that interface-id should be specified on the
        interface, which is used to receive messages from the clients, not the
        one used to forward packets to server.
\item[guess-mode] -- (scope: global, type: boolean, default: no)
        Switches relay into so called guess-mode. Under normal operation, client sends
        messages, which are encapsulated and sent to the server. During this encapsulation
        relay appends \opt{interface-id} option and expects that server will use the same
        \opt{interface-id} option in its replies. Relay then uses those \opt{interface-id}
        values to detect, which the original request came from and sends reply to the same
        interface. Unfortunately, some servers does not sent \opt{interface-id} option.
        Normally in such case, dibbler-relay drops such server messages as there is no
        easy way to determine where such messages should be relayed to. However, when
        guess-mode is enabled, dibbler-relay tries to guess the destination interface.
        Luckily, it is often trivial to guess as there are usually 2 interfaces: one
        connected to server and second connected to the clients.
\end{description}

\subsection{Relay configuration examples}
\label{example-relay}

Relay configuration file is fairly simple. Relay forwards DHCPv6
messages between interfaces. Messages from client are encapsulated and
forwarded as RELAY\_FORW messages. Replies from server are received as
RELAY\_REPL message. After decapsulation, they are being sent back to
clients.

It is vital to inform server, where this relayed message was
received. DHCPv6 does this using interface-id option. This identifier
must be unique. Otherwise relays will get confused when they will
receive reply from server. Note that this id does not need to be
alligned with system interface id (ifindex). Think about it as
"ethernet segment identifier" if you are using Ethernet network or as
"bss identifier" if you are using 802.11 network.

If you are interested in additional examples, download source version
and look at \verb+*.conf+ files.

\subsubsection{Example 1: Simple}
\label{example-relay-1}
Let's assume this case: relay has 2 interfaces: eth0 and
eth1. Clients are located on the eth1 network. Relay should receive
data on that interface using well-known ALL\_DHCP\_RELAYS\_AND\_SERVER
multicast address (ff02::1:2). Note that all clients use multicast
addresses by default. Packets received on the eth1 should be
forwarded on the eth0 interface, using multicast address. See section
\ref{example-server-relay1} for corresponding server configuration.

\begin{lstlisting}
# relay.conf
log-level 8
log-mode short
iface eth0 {
  server multicast yes
}
iface eth1 {
  client multicast yes
  interface-id 5020
}
\end{lstlisting}

\subsubsection{Example 2: Unicast/multicast}
It is possible to use unicast addresses instead/besides of default
multicast addresses. Following example allows message reception from
clients on the 2000::123 address. It is also possible to instruct
relay to send encapulated messages to the server using unicast
addresess. This feature is configured in the next section
(\ref{example-relay-multiple}).

\begin{lstlisting}
# relay.conf
log-level 8
log-mode short
iface eth0 {
  server multicast yes
}
iface eth1 {
  client multicast yes
  client unicast 2000::123
  interface-id 5020
}
\end{lstlisting}

\subsubsection{Example 3: Multiple interfaces}
\label{example-relay-multiple}
Here is another example. This time messages should be forwarded from
eth1 and eth3 to the eth0 interface (using multicast) and to the eth2
interface (using server's global address 2000::546). Also clients must
use multicasts (the default approach):

\begin{lstlisting}
# relay.conf
iface eth0 {
  server multicast yes
}
iface eth2 {
  server unicast 2000::456
}
iface eth1 {
  client multicast yes
  interface-id 1000
}
iface eth3 {
  client multicast yes
  interface-id 1001
}
\end{lstlisting}

\subsubsection{Example 4: 2 relays}
\label{example-relay-cascade}
Those two configuration files correspond to the ,,2 relays'' example
provided in section \ref{example-server-relay2}. See section
\ref{feature-relays} for detailed exmplanations.

\begin{lstlisting}
# relay.conf - relay 1
log-level 8
log-mode full

# messages will be forwarded on this interface using multicast
iface eth2 {
   server multicast yes    // relay messages on this interface to ff05::1:3
 # server unicast 6000::10 // relay messages on this interface to this global address
}

iface eth1 {
#  client multicast yes    // bind ff02::1:2
  client unicast 6011::1   // bind this address
  interface-id 6011
}
\end{lstlisting}

\begin{lstlisting}
# relay.conf - relay 2
iface eth0 {
#   server multicast yes  // relay messages on this interface to ff05::1:3
  server unicast 6011::1  // relay messages on this interface to this global address
}

# client can send messages to multicast
# (or specific link-local addr) on this link
iface eth1 {
  client multicast yes    // bind ff02::1:2
# client unicast 6021::1  // bind this address
  interface-id 6021
}
\end{lstlisting}

\subsubsection{Example 5: Guess-mode}
In the 0.6.0 release, a new feature called guess-mode has been
added. When client sends some data and relay forwards it to the
server, it always adds interface-id option to specify, which link
the data has been originally received on. Server, when responding to
such request, should include the same interface-id option in the
reply. However, in some poor implementations, server fails to do
that. When relay receives such poorly formed response from the server,
it can't decide which interface should be used to relay this
message.

Normally such packets are dropped. However, it is possible to switch
relay into a guess-mode. It tries to find any suitable interface,
which it can forward data on. It is not very reliable, but sometimes
it is better than dropping the message altogether.

\begin{lstlisting}
# relay.conf
log-level 8
log-mode short
guess-mode

iface eth0 {
  server multicast yes
}
iface eth1 {
  client multicast yes
  interface-id 5020
}
\end{lstlisting}

\subsubsection{Example 6: Relaying to multicast}
During normal operation, relay sends forwarded messages to a
\emph{All\_DHCP\_Servers} (FF05::1:3) multicast address.

Although author does not consider this an elegant solution, it is also
possible to instruct relay to forward message to a \emph{All\_DHCP\_Relay\_Agents\_and\_Servers}
(ff02::1:2) multicast address. That is quite convenient when there are several
relays connected in a cascade way (server -- relay1 -- relay2 -- clients).

For details regarding DHCPv6-related multicast addresses and relay operation, see \cite{rfc3315}.

To achieve this behavior, \emph{server unicast} can be used. Note that
name of such parameter is a bit misleading (``server unicast'' used to specify
multicast address). That parameter should be rather called ``destination address'',
but to maintain backward compatibility, it has its current name.

\begin{lstlisting}
# relay.conf
log-level 8
log-mode short

iface eth0 {
  server unicast ff02::1:2
}
iface eth1 {
  client multicast yes
  interface-id 5020
}
\end{lstlisting}

\newpage
\section{Requestor configuration}
Requestor (entity used for leasequery) does not use configuration
files. All parameters are specified by command-line switches. See
section \ref{feature-leasequery} for details.
