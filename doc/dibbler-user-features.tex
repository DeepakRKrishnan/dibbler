\newpage
\section{Features HOWTO}
This section contains information about setting up various Dibbler
features. Since this section was added recently, it is not yet
comprehensive. That is expected to change.

\subsection{Prefix delegation}
\label{feature-prefix}
Prefix delegation is a mechanism that allows two routers to delegate
(``assign'') prefixes in similar way as server can delegate
(``lease'') addresses to hosts. As specified in \cite{rfc3633}: \emph{
  The prefix delegation mechanism is intended for simple delegation of
  prefixes from a delegating router to requesting routers.  It is
  appropriate for situations in which the delegating router does not
  have knowledge about the topology of the networks to which the
  requesting router is attached, and the delegating router does not
  require other information aside from the identity of the requesting
  router to choose a prefix for delegation.  For example, these
  options would be used by a service provider to assign a prefix to a
  Customer Premise Equipment (CPE) device acting as a router between
  the subscriber's internal network and the service provider's core
  network.}

To configure server to provide prefixes, a PD pool and a client 
prefixes' length must be defined. An example section below assigns
2001:db8::/32 pool to be managed by this server. From this pool,
server will assign /48 prefixes to the clients. For example, client
can receive prefix 2001:db8:7c34::/48.

\begin{lstlisting}
pd-class {
    pd-pool 2001:db8::/32
    pd-length 48
}
\end{lstlisting}

As a general rule, server will provide random prefix to a client,
unless client provided a hint. The full prefix assignment algorithm is
as follows:

\begin{enumerate}
\item client didn't provide any hints: one prefix from each pool will
  be granted
\item client has provided hint and that is valid (supported and
  unused): requested prefix will be granted
\item client has provided hint, which belongs to supported pool, but
  this prefix is used: other prefix from that pool will be asigned
\item client has provided hint, but it is invalid (not beloninging to
  a supported pool, multicast or link-local): see point 1
\end{enumerate}

Dibbler implementation supports prefix delegation, as specified in
\cite{rfc3633}. Up to and including 0.7.3 version, client was also
capable to do non-standard tricks with delegated prefix if it was a
host, rather than router. This mode of operation was removed in
0.8.0RC1. Now client behaves the same way, regardless if it is a host
or a router. When client receives prefix on one interface (e.g. prefix
2000:1234:7c34::/48 received on eth0) it will generate subprefixes for
all other interfaces, which are up, running, non-loopback and
multicast capable. In the example depicted on
Fig. \ref{fig-prefixes-router}, received prefix was split into 3
prefixes: 2000:1234:7c34:1000::/56 for eth1, 2000:1234:7c34:2000::/56
for eth2 and 2000:1234:7c34:3000::/56 for eth3. Client support for
prefix delectation was improved in 0.8.2. Client is now able to handle
prefixes of arbitrary lengths (do not have to be divisible by 8
anymore). The only restriction is that prefix must be shorter or equal
120 bits.

It is also possible to explicitly specify which interfaces are
downlink (i.e. sub-prefixes should be assigned
to). \emph{downlink-prefix-ifaces} command may be used to disable
interface autoselection and just list downlink interfaces.

\begin{figure}[ht]
\begin{center}
\label{fig-prefixes-router}
\includegraphics[width=0.65\textwidth]{dibbler-prefixes-router}
\caption{\emph{Prefix delegation (router behaviour)}}
\end{center}
\end{figure}

It is also possible to define multiple prefix pools. See section
\ref{example-server-prefix} for simple prefix delegation configuration
for server or section \ref{example-server-prefixes} for multiple
prefixes configuration. Also section \ref{example-client-prefix}
provides information related to client configuration.

\subsection{Relays}
\label{feature-relays}
In small networks, all nodes (server, hosts and routers) are connected
to the same network segment -- usually Ethernet segment or a single
access point or hotspot. This is very convenient as all clients can
reach server directly. However, larger networks usually are connected
via routers, so direct communication is not always possible. On the
other hand it is useful to have one server, which supports multiple
links -- some connected directly and some remotely.

\begin{figure}[ht]
\begin{center}
\includegraphics[width=0.65\textwidth]{dibbler-relay}
\caption{\emph{Relay deployment}}
\end{center}
\end{figure}

Very nice feature of the relays is that they appear as actual servers
from the client's point of view. Therefore no special arrangement or
configuration on the client side is required. On the other hand, from
the administrator point of view, it is much easier to manage one DHCPv6
server and deploy several relays than manage several servers on remote
links.

It is important to understand that relays not simply forward DHCPv6
messages. Each message forwarded from client to the server is
encapsulated. Also each message forwarded from server to a client is
decapsulated. Therefore additional server configuration is required to
deal with encapsulated (i.e. relayed) traffic.

To avoid confusion during reference to a specific link (i.e. eth0 on
the relay may be different link than eth0 on the server), each link
must be referred to using its unique interface-id. It is essential to 
use the same indentifier in the relay
configuration as well as in the server, so both will refer to the same
link using the same number. See section \ref{example-server-relay1} for
example how to configure server and section \ref{example-relay-1} for
corresponding relay configuration.

\begin{figure}[ht]
\begin{center}
\includegraphics[width=0.4\textwidth]{dibbler-cascade-relays}
\caption{\emph{Cascade relays}}
\label{fig-cascade-relays}
\end{center}
\end{figure}

In larger networks it is sometimes useful to connect multiple
relays. Assuming there are 2 relays connecting server and client. Such
scenario is depicted on figure \ref{fig-cascade-relays}. Requests from
client are received by relay2, which encapsulates and sends them to
relay1. Relay1 further encapsulates those messages and sends them to
the server. Since server receives double encapsulated messages, it
must be properly configured to support such traffic. See section
\ref{example-server-relay2} for details about server configuration and
section \ref{example-relay-cascade} for example relays configuration.

\subsection{Address and prefix assignment policy}

Address and prefix assignment routines has been rewritten after 0.8.1
was released. It currently follows this algorithm:

\begin{enumerate}
\item Client classification is performed (a class is assigned to a client)
\item Client access control is performed (hosts listed on black-list are rejected,
  if any; if there is white-list defined, host must be on that list,
  otherwise it is rejected)
\item If existing lease this client/ia exists, it is assigned again
  (e.g. after host reboot)
\item Fixed lease is searched (using per-client configuration or so
  called exception mechanism). If found, this fixed lease is assigned.
\item max-client-leases is checked. If client already has maximum
  number of leases, further leases are declined.
\item Server checks if there is cached (i.e. previously assinged, but
  later released or expired) lease for this client. It is assigned, if
  possible.
\item Server checks if client sent any hints in SOLICIT or REQUEST
  message. Server tries to assign requested address or prefix. If this lease cannot be
  assinged for any reason, server tries to assign similar lease
  (i.e. from the same pool if client's hint was within supporte
  pool).
\item Otherwise, if all of above steps fails, server assigns a random
  address or prefix from supported pools.
\end{enumerate}

This algorithm is supported for non-temporary addresses and
prefixes. It is not supported for temporary addresses.

\subsection{Routing configuration}
\label{feature-routing}
Until recently, DHCPv6 protocol did not define a way to provision
routing configuration information to clients. The only way to deliver
this information to hosts was to use Router Advertisment mechanism in
Neighbor Discovery protocol \cite{rfc4862}. While that approach works,
it brings a number of drawbacks. In particular:
\begin{enumerate}
\item RA sent by router affects all hosts in a network. There is no
way to differentiate this information on a per host basis. There is no
way to define additional routing information for specified class of
hosts (e.g. one department in a corporate network).
\item RA and DHCPv6 configuration has to be consistent. That is very
doable, but somewhat problematic, because network configuration has to
be specified in several places. Moreover, it does not scale too
well. There are routers located in every segment of a network, while
there may be just a single DHCPv6 server deployed that serves many links.
\item Administrators experienced with IPv4 that are migrating their
networks to IPv6 ask this question very frequently: ``How do I
configure routing?''. Until recently the proper answer to that
question was ``you don't''.
\item In mobile environment, mobile nodes had to wait for RA and then
start DHCPv6 exchange. Although hosts can request RA by sending Router
Solicitation (RS), that may sometimes not work, as routers have upper
limits of how many RA they are allowed to sent.
\end{enumerate}

To solve aforementioned problems, a DHCPv6-based solution was
proposed \cite{draft-route-option}. It allows provisioning of IPv6
routing information. In particular, it allows configuration of a
default route, any reasonable number of specific routes and routes
available on link. This feature was introduced in Dibbler
0.8.1RC1. Both server and client support it. Dibbler sources come with
examples config files. See \verb+doc/example/server-route.conf+
and \verb+doc/example/client-route.conf+ for details. 

\Note This specification is not approved yet. It will change in the
future. In particular, IANA have not assigned specific option values
yet. Dibbler currently uses 242 for NEXT\_HOP and 243 for RTPREFIX
options. Those values will change.

\Note Current implementation is a prototype. It does support only one
route per router, only one router and only a single route
on-link. Although server is able to parse config that defines more
than one, it will provision only the first route or router information
to a client. That is implementation limitation that will be removed in
future releases. That is not a spec limitation.

To configure routing on a server side, following config may be used
\begin{lstlisting}
# Example server configuration file: server-route.conf

iface "eth0" {
# assign addresses from this pool
 class {
   pool 2000::/64
 }


# router with a single route with infinite lifetime
 next-hop 2001:db8:1::face:b00c {
     # replace this with ::/0 to configure default route
     route 2001:db8:1::/64
 }

# a single next-hop without any routes defined (i.e. default router)
# This simplified mode is recommended only in bandwidth restricted
# networks. Please use full mode instead
# next-hop 2001:db8:1::cafe

# router with 3 routes defined in different ways
 next-hop 2001:db8:1::dead:beef {
     # route may have defined a lifetime
     route 2001:db8:2::/64 lifetime 7200
     # lifetime may be infinite
     # route 2001:db8:3::/64 lifetime infinite
 }

# prefixes available on link directly, not via router
 route 2001:db8:5::/64 lifetime 3600
}
\end{lstlisting}

Support on client's side is enabled in a very simple way:
\begin{lstlisting}
# Example client configuration file: client-route.conf

# Uncomment following line to skip confirm sending (after crash or power outage)
skip-confirm

# 7 = omit debug messages
log-level 8

# Uncomment this line to run script every time response is received
script "/var/lib/dibbler/client-notify.sh"

iface "eth0" {
  ia

  option dns-server
  option domain
  routing 1
}
\end{lstlisting}

Two features should be enabled to reasonably use this
feature. \verb+routing 1+ instructs client that is should request routing
information (NEXT\_HOP and RTPREFIX options). Once such information is
sent by the server, client will execute a notify script. Client will
run defined script and pass necessary information to it. In
particular, it will set OPTION\_NEXT\_HOP, OPTION\_NEXT\_HOP\_RTPREFIX
and OPTION\_RTPREFIX variables with contents of received
option. Please see scripts/notify-scripts/client-notify.sh for example
on how to use that information to configure routing. User is also
recommended to read Section \ref{feature-script} about details of
running a script and passed variables.

\subsection{Custom options}
\label{feature-custom-options}
Dibbler is the DHCPv6 with support for a very large number of
options. However, there are always some new options that are not yet
supported. Another case is that vendors sometimes want to develop
and validate their private options before formal standarisation
process takes place. Starting with 0.8.0RC1, both client and server
are able to handle custom options. Even though author tries to
implement support for as many options as possible, there are always
cases, when that is not
enough. Some users may also test out new ideas, before thet get
standardized. Currently only several option layouts are supported, but
that list is going to be expanded. Server is able to support
following extra formats: generic (defined by hex string), IPv6
address, IPv6 address list and string (domain). To define those
options, use the following format:

\begin{lstlisting}
#server.conf
iface "eth0" {

 class {
   pool 2001:db8:1::/64
 }

 option 145 duid 01:02:a3:b4:c5:dd:ea
 option 146 address 2001:db8:1::dead:beef
 option 147 address-list 2001:db8:1::aaaa,2001:db8:1::bbbb
 option 148 string "secretlair.example.org"
}
\end{lstlisting}

Similar list can be configured for client. However, client can ask
for such custom options for testing purposes only, as mechanism for
handling those options once received is not yet implemented, as of
0.8.0RC1. Consider it experimental for the time being. Client can
request for an option using \opt{ORO} option or even send the option
in its messages.

Note that in 0.8.2 formatting of DUID-style options has
changed. ``DUID'' keyword is now required.

\begin{lstlisting}
#client.conf
iface "eth0" {
  unicast 1
  ia

  option 145 duid 01:02:a3:b4:c5:dd:ea
  option 146 address 2001:db8:1::dead:beef
  option 147 address-list 2001:db8:1::aaaa,2001:db8:1::bbbb
  option 148 string "secretlair.example.org"

  option 149 string request
  option 150 address request
  option 151 address-list request
\end{lstlisting}

A word of warning: There are no safety checks regarding option codes,
so it is possible to transmit already defined options using this
feature. Use with caution!

\subsection{DNS Update}
\label{feature-dns-update}
During normal operation, DHCPv6 client receives one or more IPv6
address(es) from DHCPv6 server. If configured to do so, it can also
receive information about DNS server addresses. As an additional
service, DNS Update (RFC2136, \cite{rfc2136}) can be performed. This
feature known as Dynamic DNS, or DNS Update, keeps the DNS entries
synced up with DHCP. When client boots, it gets its fully qualified
domain name and this name can be used to reach this particular client
by other nodes. Details of this mechanism is described
in \cite{rfc2136} and \cite{rfc4704}.

\Note In this section, we will assume that hostnames will be used from
the example.com domain and that addresses will be provided from the
2000::/64 pool.

\begin{figure}[ht]
\begin{center}
\includegraphics[width=0.65\textwidth]{dibbler-fqdn-srv-update}
\caption{\emph{DNS Update (performed by server)}}
\end{center}
\end{figure}

There are two types of the DNS Updates. First is a so called forward
resolving. It allows to change a node's name into its address,
e.g. malcolm.example.com can be translated into 2000::123. Other kind
of record, which can be updated is a so called reverse resolving. It
allows to obtain full name of a node with know address, e.g. 2000::124
can be translated into zoe.example.com.

To configure this feature, following steps must be performed:

\begin{enumerate}
\item Configure DNS server. DNS server supporting IPv6 and dynamic
  updates must be configured. One example of such server is an
  excellent \href{http://www.isc.org/software/bind}{ISC BIND}
  software. Version used during writing of this documentation was BIND
  9.7.2. It is necessary to allow listening on the IPv6 sockets and
  define that specific domain can be updated. See example below.
\item Configure Dibbler server to provide DNS server informations for
  clients. DNS Updates will be sent to the first DNS server on the
  list of available servers.
\item Configure Dibbler server to work in stateful mode, i.e. that it
  can provide addresses for the clients. This is a default mode, so
  unless configuration was altered, this step is already done. Make
  sure that there is no ,,stateless'' keyword in the
  \verb+server.conf+ file.
\item Define list of the available names in the server configuration
  file. Make sure to use fully qualified domain names
  (e.g. malcolm.example.com), not the hostnames only.
\item Configure dibbler client to request for DNS Update. Use ,,option
  fqdn'' to achieve this.
\item Server can be configured to execute
      \begin{itemize}
       \item both (AAAA and PTR) updates by itself
       \item execute PTR only by itself and let client execute AAAA
             update
       \item don't perform any updates and let client perform AAAA
             update.
      \end{itemize}
\end{enumerate}

Note that only server is allowed to perform PTR updates. After
configuration, client and/or server should log following line, which
informs that Dynamic DNS Update was completed successfully.

As of 0.8.0, both Dibbler server and client are using TCP connection
for DNS Updates. Connections are established over IPv6. There is no
support for IPv4 connections. Server uses first DNS server address
specified in \verb+dns-server+ option. It is possible to use
differentiate between DNS addresses provided to clients and the one
used for DDNS. To override DNS updates to be performed to different
address, use the following command:

\begin{lstlisting}
fqdn-ddns-address 2001:db8:1::1
\end{lstlisting}

\begin{figure}[ht]
\begin{center}
\includegraphics[width=0.65\textwidth]{dibbler-fqdn-cli-update}
\caption{\emph{DNS Update (performed by client)}}
\end{center}
\end{figure}

\subsubsection{Example BIND configuration}
Below are example configuration files for
the \href{http://www.isc.org/software/bind}{ISC BIND 9.7.2}, developed
by \href{http://www.isc.org}{Internet Systems Consortium, Inc.}.
First is a
relevant part of the /etc/bind/named.conf configuration file. Generally,
support for IPv6 in BIND is enabled (listen-on-v6) and there are two
zones added: example.com (normal domain) and
0.0.0.0.0.0.0.0.0.0.0.0.0.0.0.2.ip6.arpa (reverse
mapping). Corresponding files are stored in \verb+example.com+ and
\verb+rev-2000+ files. For details about meaning of those directives,
please consult \emph{BIND 9 Administrator Reference Manual}.

\Note Provided configuration is not safe from the security point of
view. See next subsection for details.

\begin{lstlisting}
// part of the /etc/bind/named.conf configuration file
options {
    listen-on-v6 { any; };
    listen-on    { any; };

    // other global options here
    // ...
};

zone "example.com" {
    type master;
    file "example.com";
    allow-update   { any; };
    allow-transfer { any; };
    allow-query    { any; };

    // other example.com domain-specific
    // options follow
    // ...
};

zone "0.0.0.0.0.0.0.0.0.0.0.0.0.0.0.2.ip6.arpa" {
    type master;
    file "rev-2000";
    allow-update   { any; };
    allow-transfer { any; };
    allow-query    { any; };

   // other 2000::/64 reverse domain related
   // options follow
   // ...
};
\end{lstlisting}

% \vspace{-0.3cm}
% \begin{center}
% BIND's named.conf example
% \end{center}

Below are examples of two files: forward and reverse zone. First example
presents how to configure normal domain. As an example there is entry
provided for zoe.example.com host, which has 2000::123 address. Note
that you do not have to manually configure such entries -- dibbler will
do this automatically. It was merely provided as an example, what kind
of mapping will be done in this zone.

\begin{lstlisting}
;
$ORIGIN .
$TTL 86400      ; 1 day
example.com             IN SOA  v13.klub.com.pl. root.v13.klub.com.pl. (
                                129        ; serial
                                7200       ; refresh (2 hours)
                                3600       ; retry (1 hour)
                                604800     ; expire (1 week)
                                86400      ; minimum (1 day)
                                )
                        NS      v13.klub.com.pl.
                        A       1.2.3.4
                        TXT     "Fake domain used for Dibbler tests."
$ORIGIN example.com.
$TTL 7200       ; 2 hours
zoe                     AAAA    2000::123
\end{lstlisting}

Second example presents zone file for reverse mapping. It contains
entries for a special zone called
0.0.0.0.0.0.0.0.0.0.0.0.0.0.0.2.ip6.arpa. This zone represents 2000::/64
address space. As an example there is a static entry, which maps address
2000::999 to a canonical name kaylee.example.com. Note that you do not
have to manually configure such entries -- dibbler will do this
automatically. It was merely provided as an example, what kind
of mapping will be done in this zone.

\begin{lstlisting}
; rev-2000 example file
$ORIGIN .
$TTL 259200     ; 3 days

; this line below is split in two due to page with limitation
0.0.0.0.0.0.0.0.0.0.0.0.0.0.0.2.ip6.arpa IN
      SOA 0.0.0.0.0.0.0.0.0.0.0.0.0.0.0.2.ip6.arpa. hostmaster.ep.net. (
; this line above is split in two due to page with limitation
                                200608268  ; serial
                                86400      ; refresh (1 day)
                                1800       ; retry (30 minutes)
                                172800     ; expire (2 days)
                                259200     ; minimum (3 days)
                                )
                        NS      klub.com.pl.
$ORIGIN 0.0.0.0.0.0.0.0.0.0.0.0.0.0.0.0.0.0.0.0.0.0.0.0.0.0.0.0.2.ip6.arpa.
$TTL 86200      ; 23 hours 56 minutes 40 seconds
3.2.1                   PTR     picard.example.com.

; this line below is split in two due to page with limitation
9.9.9                   PTR     kaylee.example.com.
$ORIGIN 0.0.0.0.0.0.0.0.0.0.0.0.0.0.0.2.ip6.arpa.

; example entry: 2000::999 -> troi.example.com.
; this line below is split in two due to page with limitation
9.9.9.0.0.0.0.0.0.0.0.0.0.0.0.0.0.0.0.0.0.0.0.0.0.0.0.0.0.0.0.2.ip6.arpa
      PTR troi.example.com.
; this line above is split in two due to page with limitation
\end{lstlisting}
\Note Due to page width limitation, if the example above, two lines were
split.
%% $

\subsubsection{Secure DDNS}

Earlier Dibbler versions do not provide security for DNS Updates. This
capability has been added in 0.8.3. It is possible to protect your
updates using TSIG (Transaction Signatures), defined in RFC2845
(\cite{rfc2845}). For this feature to work, your DNS server and
Dibbler server must be both configured with the same key. The first
step to use this feature is to generate a key. Currently only HMAC-MD5
keys are supported. Please ask on dibbler-devel mailing list if you
are interested in other key types. See section \ref{mailing-list} for
details. To generate HMAC-MD5 key, a \emph{dnssec-keygen} tool from
\href{http://www.isc.org/software/bind}{ISC BIND9} can be used:

\begin{lstlisting}
dnssec-keygen -a hmac-md5 -b 128 -n HOST my-ddns-secret-key
\end{lstlisting}

For ease of configuration, dibbler uses the same key syntax in its
config file as \href{http://www.isc.org/software/bind}{ISC BIND9}
does. In particular, all statements are finished with a semicolon. For
example, the minimal set of commands to configure a key look like the
following:

\begin{lstlisting}
key "DDNS\_UPDATER" {
    secret "9SYMLnjK2ohb1N/56GZ5Jg==";
    algorithm hmac-md5;
};
\end{lstlisting}

Please keep in mind that TSIG signatures are time sensitive and they
are valid only for specified amount of time. Therefore it is essential
that your Dibbler server and your DNS server have well synchronized
clocks. It is recommented to use NTP for that purpose. By default, the
signature is valid for 300 seconds. This parameter is called a
fudge. It can be modified to a different value, if needed. Shorter
value is better from the security perspective as it shortens the
window of potential replay attack. Longer values are better from the
convenience perspective, as they are more ``forgiving'' to clock
skew. The maximum allowed value here is 65535 seconds. Please note
that such a large value is not reasonable.

An example with the fudge value set to 250 is presented below:

\begin{lstlisting}
key "DDNS\_UPDATER" {
    secret "9SYMLnjK2ohb1N/56GZ5Jg==";
    algorithm hmac-md5;
    fudge 250;
};
\end{lstlisting}

Any DNS server that supports DNS Updates (\cite{rfc2136}) and TSIG
(\cite{rfc2845}) must support HMAC-MD5 signatures. Following paragraph
explains how to configure HMAC-MD5 key
for \href{http://www.isc.org/software/bind}{ISC BIND9}. There are at
least three steps that has to be done to achieve forward (AAAA) and
reverse (PTR) updates to function properly.

\begin{enumerate}
\item Add a key. Add the same key definition that was included in your
Dibbler server.conf into BIND9 configuration. Its location varies
between systems, but it often /etc/bind/named.conf. You should also
modify your zone and reverse zone to accept updates from this new
key. Make sure that you do not define \emph{fudge} parameter, as it is
not supported by BIND9. Part of the named.conf that contains related
changes looks as follows:

\begin{lstlisting}

key "DDNS\_UPDATER" {
    secret "9SYMLnjK2ohb1N/56GZ5Jg==";
    algorithm hmac-md5;
};

... (other configuration options here)

zone "example.org" {
     type master;
     file "/path/to/your/zonefile";
     allow-update { key DDNS\_UPDATER; };
     allow-query  { any; };
};

zone "0.0.0.0.1.0.0.0.8.b.d.0.1.0.0.2.ip6.arpa" {
     type master;
     file "/path/to/your/zonefile";
     allow-update { key DDNS\_UPDATER; };
     allow-query  { any; };
}

... (other zones and configuration options here)
\end{lstlisting}

In case of any problems, please refer to \emph{BIND 9 Administrator
Reference Manual}, available
on \href{http://www.isc.org/software/bind}{Internet Systems
Consortium} website.

\subsubsection{Dynamic DNS Testing and tips}
Proper configuration of the DNS Update mechanism is not an easy
task. Therefore this section provides description of several methods of
testing and tuning BIND configuration. Please review following steps
before reporting issues to the author or on the mailing list.

\begin{itemize}
\item See example server and client configuration files described in a
  sections \ref{example-client-fqdn} and \ref{example-server-fqdn}.
  Also note that Dibbler distribution should be accompanied with
  several example configuration files. Some of them include FQDN usage
  examples.
\item Make sure that unix user, which runs BIND, is able to create and
  write file example.com.jnl. When BIND is unable to create this
  journal file, it will fail to accept updates from dibbler and will
  report failure. Check BIND log files, which are usually stored in
  the \verb+/var/log/+ directory.
\item Make sure that you have routing configured properly on a host,
  which will attempt to perform DNS Update. Use ping6 command to
  verify that DNS server is reachable from this host.
\item Make sure that your DNS server is configured properly. To do so,
  you might want to use \verb+nsupdate+ tool. It is part of the BIND
  distribution, but it is sometimes distributed separated as part of
  the dnsutils package. After executing nsupdate tool, specify address
  of the DNS server (\verb+server+ command), specify update parameters
  (\verb+update+ command) and then type \verb+send+. If nsupdate
  return a command prompt, then the update was successful. Otherwise
  nsupdate will print DNS server's response, e.g. NOTAUTH of
  SRVFAIL. See below for examples of successful forward (AAAA record)
  and reverse (PTR record) updates.
\item After DNS Update is performed, DNS records can be verified using
  dig command line tool (a part of the dnsutils package). Command
  syntax is: \verb+dig @(dns-server-address) name record-type+.  In
  the following example, this query checks for name jayne.example.com
  at a server located at 2000::1 address. Record type AAAA (standard
  record for resolving name into IPv6 address) is requested. dig tool
  provides server's response in the \verb+ANSWER SECTION:+. See
  example log below.
\item In example BIND configuration above, zone transfers, queries and
  updates are allowed from anywhere. To make this configuration more
  secure, it might be a good idea to allow updates only from a certain
  range of addresses or even one (DHCPv6 server's) address only.
\end{itemize}


%%%%%%%%%%%%%%%%%%%%%%%%%%%%%%%%%%%%%%%%%%%%%%%%%%%%%%%%%%%%%%%%%%%%%%%%%%%%%%%%

To manually make AAAA record update, type:
\begin{lstlisting}
nsupdate
>server 2000::1
>update add worf.example.com 7200 IN AAAA 2000::567
>send
\end{lstlisting}

To manually make PTR record update, type:
\begin{lstlisting}
nsupdate
>server 2000::1
>update add
3.2.1.0.0.0.0.0.0.0.0.0.0.0.0.0.0.0.0.0.0.0.0.0.0.0.0.0.0.0.0.2.ip6.arpa.
86200 IN PTR picard.example.com.
>send
\end{lstlisting}

\Note Everything between "update" and "picard.example.com" must be typed in one line.

And here is an example dig session:

\begin{lstlisting}
v13:/var# dig @2000::1 jayne.example.com AAAA
; <<>> DiG 9.3.2 <<>> @2000::1 jayne.example.com AAAA
; (1 server found)
;; global options:  printcmd
;; Got answer:
;; ->>HEADER<<- opcode: QUERY, status: NOERROR, id: 33416
;; flags: qr aa rd ra; QUERY: 1, ANSWER: 1, AUTHORITY: 1, ADDITIONAL: 2

;; QUESTION SECTION:
;jayne.example.com.             IN      AAAA

;; ANSWER SECTION:
jayne.example.com.      7200    IN      AAAA    2001::e4

;; AUTHORITY SECTION:
example.com.            86400   IN      NS      v13.klub.com.pl.

;; Query time: 6 msec
;; SERVER: 2000::1#53(2000::1)
;; WHEN: Mon Jul 24 01:38:13 2006
;; MSG SIZE  rcvd: 136
\end{lstlisting}
%% >>

\subsubsection{Accepting Unknown FQDNs}
By default, server configured to support FQDN has a list of names that
are to be provided to clients. But there are use cases, when client
uses its own name and sends it to the server. So it makes sense to
sometimes allow client's own domain names. Server does not know
anything about such names, thus its nickname "Unknown FQDN".

There are several actions that server can do, when unknown FQDN is
received. To configure such support for unknown FQDNs,
\verb+accept-unknonwn-fqdn+ option can be defined on an
interface. Depending on its, value, it may bave domain name as a parameter.
For example:

\begin{lstlisting}
iface "eth0" {

# assign addresses from this class
 class {
   pool 2000::/64
 }

# provide DNS server location to the clients
# also server will use this address to perform DNS Update,
# so it must be valid and DNS server must accept DNS Updates.
 option dns-server 2000::1

# provide their domain name
 option domain example.com

# provide fully qualified domain names for clients
# note that first, second and third entry is reserved
# for a specific address or a DUID

 option fqdn 1 64
             zebuline.example.com - 2000::1,
             kael.example.com - 2000::2,
             wash.example.com - 0x0001000043ce25b40013d4024bf5,
             zoe.example.com,
             malcolm.example.com,
             kaylee.example.com,
             jayne.example.com,
             inara.example.com

# specify what to do with client's names that are not on the list
# 0 - reject
# 1 - send other name from allowed list
# 2 - accept any name client sends
# 3 - accept any name client sends, but append specified domain suffix
# 4 - ignore client's hint, generate name based on his address, append domain name

 accept-unknown-fqdn 4 foo.bar.pl

}
\end{lstlisting}

\subsection{Introduction to client classification}
\label{feature-client-class}
It is possible to define more than one address class for a single
interface. Normally, when a client asks for an address, one of the
classes is being chosen on a random basis. If not specified otherwise,
all classes have equal probability of being chosen. However there are
cases where an Administrator wants to restrict access to a given pool
or to have distinct "client classes" associated to different address
pools. For example, Computer and IP-Telephone terminals can coexist in
the same LAN ; but the Computer must belong to given class pool
meanwhile the IP-Telephone must belong to another pool.

In order to implement the Client Class Classification, you must first
create the client class and then in the class declaration, indicate
which class to be allowed or denied. This point will be discussed in
detail in next sections.

\subsubsection{Client class  declaration}
Each client class used for class / ta / pd addressing must be defined
in the server configuration file at global scope. A client-class
declaration looks like this:

\begin{lstlisting}
Client-class TelephoneClass{
        match-if ( client.vendor-spec.en == 1234567)
}
\end{lstlisting}

Where TelephoneClass denotes the name of the client class and the
(client.vendor-spec.en == 1234567) denotes the condition an incoming
message shall match to belong to the Client-Class. The supported
operator and data will be discussed in next section.


\subsubsection{Access control}
Access control is based on a per pool basis. In the client-class
declaration; you can deny or allow the client class by using the
keyword "allow" or "deny". For example, following class accepts all
clients except those belonging to the client class "TelephoneClass":

\begin{lstlisting}
class {
     2000::/64
     deny TelephoneClass
}
\end{lstlisting}

Another example. This class accepts only client belonging to the
client class "TelephoneClass".

\begin{lstlisting}
class {
     2000::/64
     allow TelephoneClass
}
\end{lstlisting}

The rule can also be applied to TA/PD declaration. Several "allow"
directives can be associated to a given pool.

\begin{lstlisting}
ta-class {
     pool 2000::/64
     deny TelephoneClass
}

pd-class {
     pd-pool 2000::/80
     pd-length 96
     deny TelephoneClass
}
\end{lstlisting}

\subsubsection{Assigning clients to defined classes}
Classifying operators are used for assigning client to a specific class.
Currently, Dibbler supports the following Operators for classifying clients:

\begin{lstlisting}
Equal operator
    Syntax : ( Expr1 == Expr2 )
    Scope : global
    Purpose : returns "true" if Expr1 equals Expr2

And Operator
    Syntax : ( Condition1 and Condition2 )
    Scope : global
    Purpose : returns "true" if both Condition1 and Condition2 are "true"

Or operator
    Syntax : ( Condition1  or  Condition2 )
    Scope : global
    Purpose : returns "true" if either Condition1 or Condition2 is "true"

Contain Operator
    Syntax : ( String1 contain String2 )
    Scope : global
    Purpose :  returns "true" if String2 is a substring of String1

Substring Operator
    Syntax substring ( Expr1, index, length )
    Scope : global
    Purpose : returns the substring of the result of that evaluation
    that starts index characters from the beginning, continuing for
    length characters.
\end{lstlisting}

Dibbler accepts different data expressions -- or variables -- which
reflect value of options found in the packet to which the server is
responding.

\begin{description}
\item[client.vendor-spec.en] the enterprise number value of
  OptionVendorSpecific (OPTION\_VENDOR\_OPTS, option value equals to 17
  as per RFC3315)
\item[client.vendor-spec.data] the data of OptionVendorSpecific
  (OPTION\_VENDOR\_OPTS, option value equals to 17 as per RFC3315)
\item[client.vendor-class.en] the enterprise number value of
  OptionVendorClass (OPTION\_VENDOR\_CLASS, option value equal to 16 as
  per RFC3315)
\item[client.vendor-class.data] the data of OptionVendorClass
  (OPTION\_VENDOR\_CLASS, option value equals to 16 as per RFC3315)
\end{description}

\subsubsection{Examples of Client-Class Classifying}

Example 1 :
\begin{lstlisting}
Client-class CPEClass {
        match-if ( client.vendor-spec.data contain CPE )
}
\end{lstlisting}
Client belongs to CPEClass if its request message contains the Vendor
Specific option with the data field including the substring "CPE".


Example 2 : Combination with AND operator
\begin{lstlisting}
Client-class TelephoneClass {
  match-if (( client.vendor-spec.en == 1234) and ( client.vendor-spec.data contain CPE ) )
}
\end{lstlisting}

Example 3 : Combination with OR operator
\begin{lstlisting}
Client-class TelephoneClass {
  match-if (( client.vendor-spec.en == 1234) or ( client.vendor-spec.data contain CPE ) )
}
\end{lstlisting}

\subsection{External script}
\label{feature-script}

\Note Support for external scripts (often called \emph{notify script}
  was rewritten in 0.8.1RC1 release. Note that mapping prefix and
notify scripts were removed. Support for server-side script was
introduced in 0.8.1RC1.

Dibbler-client is able to receive addresses, prefixes and numerous
additional options. It will do its best to set up those parameters in
the system. However, the need for some extra processing may arise. The
most elegant solution is to call external script every time the
configuration changes. Dibbler client may be configured to call
external script every time REPLY is received for REQUEST (new
parameters added), RENEW (parameters were updated) or RELEASE
(parameters were deleted).

Name of this script is specified using \verb+script+ keyword followed
by absolute path to script. Script will be called with a single
parameter, denoting current operation. Its value will be one of
``add'', ``update'', ``delete'' or ``expire''. Currently ``expire''
event is triggered on server-side only. \footnote{Please send your
feedback to mailing list if you need it also on client-side.} Actual
values of received parameters are passed as environment variables. In
particular, IFNAME and IFINDEX variables denote interface name and
interface index that was used to communicate with server,
respectively. Another essential variable set is REMOTE\_ADDR. It
defines address from which packet originated. That is client's address
(when run on server) and server's address (when run on
client). Client's message type is passed in CLNT\_MESSAGE
variable. Server's response is passed in SRV\_MESSAGE. Note that
server's reply is most often REPLY as script execution is skipped after
sending ADVERTISE.

Addresses are passed in variables ADDR1, ADDR2 and following. Note
that each ADDR variable is accompanied with two additional variables:
ADDR1PREF (address preferred lifetime) and ADDR1VALID (address valid
lifetime). Prefixes are passed in variables PREFIX1, PREFIX2 and
following. Note that each PREFIX variable is accompanied with three
additional variales: PREFIX1LEN (prefix length), PREFIX1PREF (prefix
preferred lifetime), and PREFIX1VALID (prefix valid lifetime). Support
for additional options is in progress. Options are passed as
environment variables. For example client DUID (conveyed in option
code 1), will be passed as OPTION1.

To enable script execution, \verb+script+ global option must
be added to \verb+client.conf+ file. For example:

\begin{lstlisting}
# client.conf
script /var/lib/dibbler/script.sh

iface eth0 {
   ia
}
\end{lstlisting}

\subsection{Confirm}
\label{feature-confirm}
Client detects if previous client instance was not shutdown properly
(due to power outage, client crash, forceful shutdown or similar
event). In such case, it reads existing address database and checks if
assigned addresses may still be valid. If that is so, it tries to
confirm those addresses by using \msg{CONFIRM} message.

If you want to provoke this kind of scenario on purpose, you can run
dibbler-client normally, then forcefully kill the procss (by sending
kill -9 signal, or pressing ctrl-\\ under Linux). Make sure that you
rerun client before address valid lifetime expires.

Currently, client does support only IAs in the \msg{CONFIRM}.

\subsection{Mobility}
Client can also be compiled with support for link change detection.
The intended use for this feature is mobility. Client is able to
detect when it moves to new link and react accordingly. Client sends
\msg{CONFIRM} message to verify that its currently held address is
still usable on this new link.

\subsection{Leasequery}
\label{feature-leasequery}
Servers provide addresses, prefixes and other configuration options to
the clients. Sometime administrators may want to obtain information
regarding certain leases, e.g. who has been given a specific address
or what addresses have been assigned to a specific client. This
mechanism is called Leasequery \cite{rfc5007}. New DHCPv6 participant
called requestor has been defined. Its sole purpose is to send queries
and receive responses. Dibbler provides example implementation. To
define a query, command line parameters are used.

There are two types of queries: by address ("who leases this address?")
and by client identifier ("what addresses has this client?"). To specify
one of such types, \verb+-addr+ or \verb+duid+ command-line switches
can be used. It is also mandatory to specify (using \verb+-i IFACE+),
which interface should be used to transmit the query.

Here is a complete list of all command-line switches:

\begin{description}
\item[-i IFACE] -- defines thru which interface should the query be sent
\item[-addr ADDR] -- sets query type to query by address. Also defines
  address, which the query will be about.
\item[-duid DUID] -- sets query type to query by client
  indentifier. Also defines client intentifier.
\item[-timeout SECS] -- specifies time, which requestor should wait
  for response.
\item[-dstaddr ADDR] -- destination address of the lease query
  message. By default messages are sent to the multicast address
  (ff02::1:2). To transmit query to an unicast addres, use this option.
\end{description}

Example query 1: Who has 2000::1 address?

\begin{lstlisting}
dibbler-requestor -i eth0 -addr 2000::1
\end{lstlisting}

Example query 2: Which addresses are assigned to client with specific
client identifier?

\begin{lstlisting}
dibbler-requestor -i eth0 -duid 00:01:00:01:0e:8d:a2:d7:00:08:54:04:a3:24
\end{lstlisting}

\subsection{Stateless vs stateful and IA, TA options}
\label{feature-stateless-stateful}
This section explains the difference between stateless and stateful
configurations. IA and TA options usage is also described.

Usually, normal stateful configuration based on non-temporary
addresses should be used. If you don't know, what temporary addresses
are, you don't need them.

Note that DHCPv6 stateless autoconfiguration is part of stateless autoconfiguration
defined in \cite{rfc4862}.

There are two kinds of configurations in DHCPv6 (\cite{rfc3315},
\cite{rfc3736}):
\begin{description}
  \item[stateful] -- it assumes that addresses (and possibly other
    parameters) are assigned to a client. To perform this kind of
    configuration, four messages are exchanged: \msg{SOLICIT},
    \msg{ADVERTISE}, \msg{REQUEST} and \msg{REPLY}.
  \item[stateless] -- when only parameters are configured (without
    assigning addresses to a client). During execution of this type of
    configuration, only two messages are exchanged: \msg{INF-REQUEST}
    and \msg{REPLY}.
\end{description}

During normal operation, client works in a stateful mode. If not
instructed otherwise, it will request one or more normal
(i.e. non-temporary) address. It will use \opt{IA} option (Identity
Association for Non-temporary Addresses, see \cite{rfc3315} for
details) to request and retrieve addresses. Since this is a default
behavior, it does not have to be mentioned in the client configuration
file. Nevertheless, it can be provided:

\begin{lstlisting}
# client.conf
iface eth0 {
  ia
  option dns-server
}
\end{lstlisting}

In a specific circustances, client might be interested in obtaining
only temporary addresses. Although this is still a stateful mode, its
configuration is sligtly different. There is a special option called
\opt{TA} (Identity Association for Temporary Addresses, see
\cite{rfc3315} for details). This option will be used to request and
receive temporary addresses from the client. To force client to
request temporary addresses instead of permanent ones, \verb+ta+
keyword must be used in client.conf file. If this option is defined,
only temporary address will be requested. Keep in mind that temporary
addresses are not renewed.

\begin{lstlisting}
# client.conf
iface eth0 {
  ta
  option dns-server
}
\end{lstlisting}

It is also possible to instruct client to work in a stateless mode. It
will not ask for any type of addresses, but will ask for specific
non-adress related configuration parameters, e.g. DNS Servers
information. This can be achieved by using \verb+stateless+
keyword. Since this is a global parameter, it is not defined on any
interface, but as a global option.

\begin{lstlisting}
# client.conf
stateless
iface eth0
{
  option dns-server
}
\end{lstlisting}

Some of the cases mentioned above can be used together. However,
several combinations are illegal. Here is a complete list:

\begin{description}
\item[none] -- When no option is specified, client will assume one IA
  with one address should be requested. Client will send \verb+ia+
  option (stateful autoconfiguration).
\item[ia] -- Client will send \verb+ia+ option (stateful
  autoconfiguration).
\item[ia,ta] -- When both options are specified, client will request
  for both - Non-temporary as well as Temporary addresses (stateful
  autoconfiguration).
\item[stateless] -- Client will request additional configration
  parameters only and will not ask for addresses (stateless
  autoconfiguration).
\item[stateless,ia] -- This combination is not allowed.
\item[stateless,ta] -- This combination is not allowed.
\item[stateless,ia,ta] -- This combination is not allowed.
\end{description}

\subsection{Server address caching}
Previous Dibbler versions assigned a random address from the available
address pool, so the same client received different address each time it
asked for one. In the 0.5.0 release, new mechanism was introduced
to make sure that the same client gets the same address each time. It is
called \emph{Server caching}.

Below is the algorithm used by the server to assign an address to the client.

\begin{itemize}
 \item if the client provided hint, it is valid (i.e. is part of the
       supported address pool) and not used, then assign requested address.
 \item if the client provided hint, it is valid (i.e. is part of the
       supported address pool) but used, then assign free address from
       the same pool.
 \item if the client provided hint, but it is not valid (i.e. is not
       part of the supported address pool, is link-local or a multicast
       address), then ignore the hint completety.
 \item if the did not provide valid hint (or provided invalid one), try
       to assign address previously assigned to this client (address caching)
 \item if this is the first time the client is seen, assign any address
       available.
\end{itemize}
%% see SrvOptions/SrvOptIA_NA.cpp, TSrvOptIA_NA::getFreeAddr() method


\subsection{XML files}
\label{feature-xml}
During its execution, all dibbler components (client, server and
relay) store its internal information in the XML files. In Linux
systems, they are stored in the \verb+/var/lib/dibbler+ directory. In
Windows, current directory (i.e. directory where exe files are
located) is used instead. There are several xml files generated. Since
they are similar for each component, following list provides
description for server only:

\begin{itemize}
\item server-CfgMgr.xml -- Represents information read from a
  configuration file, e.g. available address pool or DNS server
      configuration.
\item server-IfaceMgr.xml -- Represens detected interfaces in the
  operating system, as well as bound sockets and similar information.
\item server-AddrMgr.xml -- This is database, which contains identity
  associations with associated addresses.
 \item server-cache.xml -- Since caching is implemented by the server
      only, this file is only created by the server. It contains
      information about previously assigned addresses.
\end{itemize}


\subsection{Authentication and Authorization}
\label{feature-auth}

Implementation of authentication and authorization in Dibbler is
loosely based on \cite{draft-aaa}. Mainly option formats have been
used, for interoperability purposes. This means that there should be
no reasonable expectation of interoperability with other
implementations.

Draft does not specify how to communicate with home and foreign AAA
servers (AAAH and AAAF) using Diameter or Radius protocol, so Dibbler
uses a different, simpler approach. Keys are stored locally in
files. (see Fig. \ref{fig-aaa}).

\begin{figure}[ht]
\begin{center}
\label{fig-aaa}
\includegraphics[width=0.65\textwidth]{dibbler-aaa}
\caption{\emph{Simplified model of AAA}}
\end{center}
\end{figure}

For each pair of client and server three files are needed. Client uses
a file \texttt{AAA-SPI}, which contains 32-bit AAA-SPI (AAA Security
Parameter Index) --- eight hexadecimal digits, to properly introduce
himself (authorize) to server. Also it needs file named
\texttt{AAA-key-\textit{AAASPI}}, which contains a key that is used to
generate authentication information in AAAAUTH and AUTH options. The
AAA-key is any number of arbitrary chosen bytes and is generated by
administrator of DHCPv6 server. The server needs only one file per
client to properly communicate using authentication. The file is named
\texttt{AAA-key-\textit{AAASPI}}, where \textit{AAASPI} is the same
value, that client has in \texttt{AAA-SPI} file. This file contains
the same AAA-key, that client has in \texttt{AAA-key} file. Dibbler
searches for those files in \textit{AAA directory}, which is
\texttt{/var/lib/dibbler/AAA} when running under Linux and current
directory, when running under Windows.

Typical scenario of preparing a client and server to use authentication:
\begin{enumerate}
 \item Administrator generates \texttt{AAA-key-\textit{AAASPI}}
   file. \textit{AAASPI} is an arbitrary chosen 32-bit number (as
   described above). The file contains any AAA-key and can be
   administrator's favorite poem or can be simply generated using
   \texttt{dd} and \texttt{/dev/urandom}:
\begin{lstlisting}
$ dd if=/dev/urandom of=AAA-key-b9a6452c bs=1 count=32
\end{lstlisting}
%%$

\item Administrator creates file \texttt{AAA-SPI} which contains
  previously chosen \textit{AAASPI}. This file will be used by the
  client only.

\item Administrator transfers \texttt{AAA-SPI} and
  \texttt{AAA-key-\textit{AAASPI}} to the client, using some secure
  method (e.g. mail+PGP, scp, https) to avoid sniffing the key by a
  potential attacker.

\item Client: User stores \texttt{AAA-SPI} and
  \texttt{AAA-key-\textit{AAASPI}} in \textit{AAA directory}.

\item Server: Administrator stores \texttt{AAA-key-\textit{AAASPI}} in
  \textit{AAA directory}.

\end{enumerate}

For example, configuration files can look like this:

\begin{itemize}
\item Server's \texttt{AAA-key-b9a6452c} and client's \texttt{AAA-key}
  (32 bytes):
\begin{lstlisting}
ma8s9849pujhaw09y4h[80pashydp80f
\end{lstlisting}

\item Client's \texttt{AAA-SPI} (8 bytes):
\begin{lstlisting}
b9a6452c
\end{lstlisting}
\end{itemize}

When configuration files are prepared and stored in client's and
server's \textit{AAA directory} you are ready to use
authentication. For detailed description of possible options see
\ref{client-conf-reference}. For quick start:
\begin{itemize}
 \item set ``\texttt{auth-enabled true}'' in \texttt{client.conf}

 \item set ``\texttt{auth-method digest-hmac-sha256}'' in
   \texttt{server.conf}

\end{itemize}

See section \ref{example-client-auth} for example client configuration
and \ref{example-server-auth} for server configuration.

\subsection{Exceptions: per client configuration}
\label{feature-exceptions}
All configuration parameters (except FQDN) are the same for all
clients, e.g. all clients will receive the same domain name and the
same DNS servers information.

However, it is sometimes useful to provide some clients with different
configuration parameters. For example computers from the accouting
department in a corporate network may be configured to be in a
different subdomain. Is is possible to specify that for particular
client different configuration options should be provided. Each client
is identified by its DUID. This mechanism is called \emph{per client
  configuration}, but it is sometimes referred to as
\emph{exceptions}. Support for per client prefix configuration has
been added in 0.8.2RC1.

See section \ref{example-server-exceptions} for server configuration
examples.

\subsection{Vendor specific information}
\label{feature-vendor-spec}
Dibbler supports vendor specific information options. As the name
suggests, that option is specific to a particular vendor. To be able
to support any vendor in a flexible manner, values are specified in a
hex format in \verb+server.conf+. For example:

\begin{lstlisting}
 option vendor-spec 1234-0x00002fedc
\end{lstlisting}

When client asks for a vendor-specific info, server will send
vendor-specific info option with enterprise number set to 1234 and
value option-data will be 00002fedc.

Although uncommon, it is also possible to specify multiple vendor
options. Another \verb+server.conf+ example:

\begin{lstlisting}
 option vendor-spec 1234-0x00002fedc,5678-0x0002aaaa
\end{lstlisting}

Server algorithm for choosing, which vendor option should be sent,
works as follows:

\begin{itemize}
\item When client requests for a speficic vendor (i.e. sends
  \opt{vendor-spec info} option with vendor field set), it will
  receive option for that specific vendor (i.e. requested 1234, got
  1234).
 \item When client requests any vendor (i.e. sends only \opt{option
   request} option with vendor-spec mentioned), it will receive first
   \opt{vendor-spec info} option from the list (i.e. 5678/0002aaaa).
 \item When client requests for not supported vendor (i.e. 11111), it
   will receive first vendor-spec option from the list
   (i.e. 5678/0002aaaa).
\end{itemize}

It is possible to configure Dibbler client to ask for vendor-specific
info. Granted value will not be used, so from the client's point of
view this feature may be used as testing tool for the server. Client
can request \opt{vendor-specific information} option in one of the
following ways:

\begin{description}
\item[option vendor-spec] -- Only \opt{option request} option will be
  sent with \opt{vendor-spec info} option mentioned.
\item[option vendor-spec 1234] -- \opt{option request} option will be
  sent with \opt{vendor-spec info} option mentioned, but also
  \opt{vendor-spec info} option with enterprise number set to 1234
  will be sent.
\item[option vendor-spec 1234 0x0a0b0c0d] -- \opt{option request}
  option will be sent with \opt{vendor-spec info} option mentioned,
  but also \opt{vendor-spec info} option with enterprise number set to
  1234 and option-data will be sent.
\end{description}

Although that is almost never needed, it is possible to configure
client to request multiple vendor-specific options at the same
time. That is also supported by the server. See
\ref{example-client-vendor-spec} for examples.


However, if client sends requests for multiple vendor-specific
options, which are not supported by the server, for each sent option,
server will assign one default vendor-spec option.

See \ref{example-client-vendor-spec} for client example and
\ref{example-server-vendor-spec} for server examples.

\subsection{Not connected interfaces (inactive-mode)}
\label{feature-inactive-mode}
During normal startup, client tries to bind all interfaces defined in
a configuration file. If such attempt fails, client reports an error
and gives up. Usually that is best action. However, in some cases it
is possible that interface is not ready yet, e.g. WLAN interface did
not complete association. Dibbler attempt to detect link-local
addresses, bind any sockets or initiate any kind of communication will
fail. To work around this disadvantage, a new mode has been
introduced in the 0.6.0RC4 version. It is possible to modify client
behavior, so it will accept downed and not running interfaces. To do
so, \emph{inactive-mode} keyword must be added to client.conf file. In
this mode, client will accept inactive interfaces, will add them to
inactive list and will periodically monitor its state. When the
interface finally goes on-line, client will try to configure it.

To test this mode, you can simulate deassociation using normal
Ethernet interface. Issue following commands:

\begin{itemize}
\item Bring down your interface (e.g. ifconfig eth0 down)
\item edit \verb+client.conf+ to enable inactive-mode
\item execute client: \verb+dibbler-client run+
\item client will print information related to not ready interface,
  and will periodically (once in 3 seconds) check interface state.
\item in a separate console, issue \verb+ifconfig eth0 up+ to bring
  the interface up.
\item dibbler-client will detect this and will initiate normal
  configuration process.
\end{itemize}

In the 0.6.1 version, similar feature has been introduced on the
server side. See sections \ref{example-client-inactivemode} and
\ref{example-server-inactivemode} for configuration examples.

\subsection{Parameters not supported by server (insist-mode)}
\label{feature-insist-mode}

Client can be instructed to obtain several configuration options, for
example DNS server configuration or domain name. It is possible that
server will not provide all requested options. Older versions of the
dibbler client had been very aggressive in such case. It tried very
hard to obtain such options. To do so, it did send \msg{INF-REQUEST}
to obtain such option. It is possible that some other DHCPv6 servers
will receive this message and will reply with valid configuration
parameters. This behavior has changed in the 0.6.0RC4 release. Right
now when client does not receive all requested options, it will
complain, but will take no action. To enable old behavior, so called
insist-mode has been added. To enable this mode, add
\verb+insist-mode+ at the global section of the \verb+client.conf+
file. Example configuration file is provided in the
\ref{example-client-insistmode}.

\subsection{Different DUID types}
\label{feature-duid-types}
There are 3 different types of the DUID (DHCP Unique Identifier):
\begin{itemize}
\item type 1 (link-layer + time) -- this DUID is based on Link-layer
  address and a current timestamp. According to spec \cite{rfc3315},
  that is a default type.
\item type 2 (enterprise number) -- this DUID is based on the Private
  Enterprise Number assigned to larger companies. Each vendor should
  maintain its own space of unique identifiers.
\item type 3 (link-layer) -- this DUID is based on link-layer address
  only.
\end{itemize}

According to spec \cite{rfc3315}, it is recommended to use link-layer
+ time, if possible. That DUID type provides most uniqueness. It has
one major drawback -- it is impossible to know DUID before it is
actually generated. That poses significant disadvantage to sysadmins,
who want to specify different configuration for each client. In such
cases, it is recommended to switch to link-layer only (type 3) DUIDs.

During first executing dibbler-client will generate its DUID and store
it in \verb+client-duid+ file on disk. During next startup DUID will
be read from the file, not generated.

It is possible to specify, what DUID format should be used. It is
worth noting that such definition is taken into consideration during
DUID generation only, i.e. during first client execution. To specify
DUID type, put only one of the following lines in the
\verb+client.conf+ file:

\begin{lstlisting}

# uncommend only ONE of the lines below
duid-type duid-llt
#duid-type duid-en 1234 0x56789abcde
#duid-type duid-ll

iface eth0 {
   ia
   option dns-server
}
\end{lstlisting}

When using link-layer+time or link-layer DUID types, dibbler will
autodetect addresses. To generate enterprise number-based DUID,
specific data must be provided: enterprise-number (a 32-bit integer,
1234 in the example above) and a enterprise-specific indentifier of
arbitrary length (56:78:9a:bc:de in the example above).

\subsection{Debugging/compatibility features}
During interoperability test session, it has been discovered that
sometimes various different implementations of the DHCPv6 protocol has
problem to interact with each other. As the protocol itself does not
specify all aspects and details, some things can ba done differently
and there is no only one ,,proper way''. It also happens that some
implementations may have problems with different than its authors
expected behaviors. To allow better interoperation between such
implementation, dibbler has some features, which cause different
behaviors. This could result in a successful operation with other
servers, clients and relays.

Normal users don't have to worry about those options, unless they are
using different servers, clients and relays. Those options also may be
useful for other vendors, who want to test their
implementations. Therefore those options can be perceived as a
debugging or testing features.

\subsubsection{Interface-id option}
During message relaying (done by relays), options can be placed in the
\msg{RELAY-FORW} message is arbitrary order. In general, there are two
options used: \opt{interface-id} option and \opt{relay-message}
option. The former defines interface identifier, which the original
data has been received from, while the later contains the whole
original message. When several relays are used, such message-in-option
encapsulation can occur multiple times.

It is possible to instruct relay to store \opt{interface-id} before
\opt{relay-message} option or after. There is also possibility to
instruct server to omit the \opt{interface-id} option altogether, but
since this violates \cite{rfc3315}, it should not be used. In general,
this configuration parameter is only useful when dealing with buggy
relays, which can't handle all option orders properly. Consider this
parameter a debugging feature.

Similar parameter is defined for the server. Server uses it during
\msg{RELAY-REPL} generation.

See description of the \emph{interface-id-order} parameters in Server
configuation (section \ref{server-conf}) and Relay configuration
(section \ref{relay-conf}).

\subsubsection{Non-empty IA\_NA option}
When client is interested in receiving an address, it sends
\opt{IA\_NA} option. In this option it may (but don't have to) include
addresses (using \opt{IAADDR} suboption) as hints for the server.

It has been detected that some servers does not support properly
(perfectly valid) empty \opt{IA\_NA} options. To work around this
problem, dibbler-client can be instructed to include two \opt{IAADDR}
in the \opt{IA\_NA} option. Here is minimal example config, which
achieves that:

\begin{lstlisting}
iface eth0 {
  ia {
     address
     address
  }
}
\end{lstlisting}

\subsubsection{Providing address/prefix hints}
Dibbler client can be instructed to send specific addresses or
prefixes in its \msg{SOLICIT} messages. This can be achieved by using
following syntax:
\begin{lstlisting}
# client.conf - request specific address/prefix
iface eth0 {
    ia {
        address { 2001:db8:dead:beef:: }
    }
    pd {
       prefix 2001:db8:aaaa::/64
    }
}
\end{lstlisting}

Be default, client will use those addresses in \msg{SOLICIT} message
only. When transmitting \msg{REQUEST} message, it will copy proposals
from \msg{ADVERTISE} message, received from a server. To force client
to use those specified addresses and/or prefixes also in
\msg{REQUEST}, please use \verb+insist-mode+ directive.

\subsection{Experimental features}

This section contains experimental features. Besides serving as a
general purpose DHCPv6 solution, dibbler is also used as a research
tool for new ideas. \footnote{This was particularly true during my
  Ph. D. research.} Normal users are recommended NOT to use any
of those features. Advanced users should take extra caution. Also be
aware that those options may not work as expected, may be incomplete
and not documented properly. You have been warned.

Since those mechanisms are non-standard, they are disabled by
default. To enable them, ,,experimental'' keyword must be placed in
the \verb+client.conf+ or \verb+server.conf+ files.

\subsubsection{Address Parameters}
\label{feature-addr-params}
\textbf{Note: This feature is experimental, i.e. it is not described
by any RFC or even internet draft. Don't use it, unless you exactly
know what you are doing.}

There is ongoing process to register and publish internet draft,
which describes this operation. Latest versions of this draft will be
availabe at \url{http://klub.com.pl/dhcpv6/doc/}.

RFC3315 (\cite{rfc3315}) defines means of allocating IPv6 addresses to
all interested clients. Clients are able to obtains IPv6 addresses and
other configuration parameters from the servers. Unfortunately, client
after obtaining an address, are not able to communicate each other due
to missing prefix information. That property of the DHCPv6 procotol is
sometimes perceived as a major disadvantage. To overcome this
deficiency, an extension to the protocol has been proposed.

It is possible to attach additional option conveyed in normal IAADDR
option. That additional option, called ADDRPARAMS option, contains
additional information related to that address. To maintain backward
compatibility, server does not send such option by default, even when
configured to support it. To make server send this option, client must
explicitly ask for it.

Below are example configuration files for server and client. Note that
since that is an non-standard feature, user must explicitly allow
experimental options before configuring it (thus ,,experimental''
keyword is required).

Example \verb+client.conf+ configuration file:

\begin{lstlisting}
#client.conf
log-mode short
log-level 8


iface "eth0" {
  ia {
     addr-params
  }
}
\end{lstlisting}

Example \verb+server.conf+ configuration file:

\begin{lstlisting}
#server.conf
log-level 8

experimental
log-mode short

iface eth0 {

 t1 60
 t2 96
 prefered-lifetime 120
 valid-lifetime 180

 class {
   addr-params 80
   pool 2001:458:ff01:ff03::/80
 }
}
\end{lstlisting}

\subsubsection{Remote Autoconfiguration}
\label{feature-remote-autoconf}
Every time a node attaches to a new link, it must renew or
obtain new address and parameters, using DHCPv6 protocol (namely
\msg{CONFIRM} or \msg{SOLICIT} messages.  In case of mobile nodes,
it is beneficial to obtain address and other  configuration parameters
remotely, before actually attaching to destination link.  This
extension provides experimental support for such operation.
Details of this mechanism are thoroughly discussed in \cite{phd,
  draft-remote-autoconf, networks2010, atnac2010}.

The idea is that once client attaches to its current location, normal
configuration procedure is initiated (\msg{SOLICIT}, \msg{ADVERTISE},
\msg{REQUEST} and \msg{REPLY}). However, besides requesting the usual
options, client also asks for \opt{NEIGHBORS} option. Server provides
that option that contains list of available DHCPv6 servers at
neighboring networks.

Once client gains that information, it then initiates remote
autoconfiguration process, i.e. it sends \msg{SOLICIT} message to each
of the newly discovered neighbors, requesting single IPv6
address. Servers respond remotely, using \msg{REPLY} message. Once
this exchange is completed, client knows its new IPv6 address for each
of the potential handover targets. What is especially important is
that client obtains that knowledge, while still being connected to old
location. It may leverage that knowledge, e.g. to update his
correspondent nodes in advance.

As Dibbler client is not a mobility software itself, it has to
communicate with Mobile IPv6 stack somehow. Therefore it triggers
./remote-autoconf script every time remote autoconfiguration is
concluded.

Note that to support this scenario, both client and all participating
servers must have unicast and rapid-commit support enabled.

Following series of server.conf files demonstrate, how 3 servers can
be configured to incorm client about their 2 neighbors.

\begin{lstlisting}
#server.conf for server1.
log-level 8
log-mode short
preference 2

experimental

iface "eth0" {

 t1 1800
 class {
   pool 2001:db8:1111::/64
 }

 rapid-commit 1
 unicast 2001:db8:1111::f

 option neighbors 2001:db8:2222::f,2001:db8:3333::f
}
\end{lstlisting}

\begin{lstlisting}
#server.conf for server2
log-level 8
log-mode short
preference 1

experimental

iface "eth1" {
 unicast 2001:db8:2222::f
 rapid-commit 1

 class {
   pool 2001:db8:2222::/64
 }


 option neighbors 2001:db8:1111::f,2001:db8:3333::f
}
\end{lstlisting}

\begin{lstlisting}
log-level 8
preference 0
experimental

iface "eth1" {

 unicast 2001:db8:3333::f
 rapid-commit 1
 class {
   pool 2001:db8:3333::/64
 }

 option dns-server 2001:db8:3333::f
 option neighbors 2001:db8:1111::f,2001:db8:2222::f
}
\end{lstlisting}

Client also needs to have enabled number of features. Following config
file may serve as an example:

\begin{lstlisting}
log-mode short
log-level 8

experimental
remote-autoconf

iface "eth0" {
  ia
  unicast 1
}
\end{lstlisting}

\subsection{Obsoleted experimental features}
This subsection describes experimental features that are not supported
anymore. This list is provided for historical reasons. It may be
useful for someone to ease tracking of features removal, e.g. to
get the latest version that still has support for something.

\subsubsection{Mapping prefix}
Mapping prefix was an extension that altered client's behavior when
delegated prefix is received. Instead of considering it as a prefix
that should be distributed on other interfaces, it is used as a mapping
prefix. Normal prefix processing is supressed and external script is
executed: \verb+mappingprefixadd+ or \verb+mappingprefixdel+. That
script must be present in the working directory (that would be
\verb+/var/lib/dibbler+ under Linux or current directory (Windows).
This feature was removed in 0.8.0RC1.

\subsubsection{Tunnel mode}
As support for DS-Lite \cite{rfc6334} support was added in
0.8.0RC1, the old support for configuring tunnels was removed.
