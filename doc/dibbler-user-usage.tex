%%
%% Dibbler - a portable DHCPv6
%%
%% authors: Tomasz Mrugalski <thomson@klub.com.pl>
%%
%% released under GNU GPL v2 licence

\newpage
\section{Installation and usage}
\label{install}
Client, server and relay are installed in the same way. Installation
method is different in Windows and Linux systems, so each system
installation is described separately. To simplify installation, it
assumes that binary versions are used\footnote{Compilation is not
required, usually binary version can be used. Compilation should be
performed by advanced users only, see Section \ref{compile} for
details.}.

\subsection{Linux installation}
Starting with 0.4.0, Dibbler consists of 3 different elements: client,
server and relay. During writing this documentation, Dibbler is already
part of many Linux distributions. In particular:
\begin{description}
 \item[\href{http://debian.org}{Debian
 GNU/Linux}, \href{http://ubuntu.com}{Ubuntu}] and derived -- use
 standard tools (apt-get, aptitude and similar) to install
 dibbler-client, dibbler-server, dibbler-relay or dibbler-doc packages.
\item[\href{http://opensuse.org}{OpenSUSE}] -- use standard
 installation mechanism.
\item[\href{http://www.pld-linux.org}{PLD GNU/Linux}]
 -- use standard PLD's poldek tool to install dibbler
 package.  \item[\href{http://www.gentoo.org}{Gentoo Linux}] -- use
 emerge to install dibbler (e.g. emerge dibbler).
\item[\href{http://openwrt.org}{OpenWRT}] -- there are package
 definitions for OpenWRT. At time of this writing, they were very
 outdated (using 0.5.0 version).
\end{description}

If you are using other Linux distribution, check out if it already
provides Dibbler packages. You may use them or compile the sources on
your own. See Section \ref{compile} for details regarding compilation
process. Dibbler used to provide native DEB and RPM packages, but due
to limited resources, author is not continuing this activity. If you
are a Dibbler package maintainer and want your package to be put on
dibbler website, please send such request on mailing list (see
Section \ref{mailing-list}).

To install Dibbler on Debian or other system that provides apt-get
package management system, run \verb+apt-get install package+
command. For example, to install server and client, issue the
following command:

\begin{lstlisting}
apt-get install dibbler-server dibbler-client
\end{lstlisting}

To install Dibbler in Gentoo systems, just type:
\begin{lstlisting}
emerge dibbler
\end{lstlisting}

\subsection{Windows installation}
Dibbler supports Windows XP and 2003 since the 0.2.1-RC1 release.
Support for Vista was added somewhere around 0.7.x. Support for
Windows 7 was added in 0.8.0RC1. In version 0.4.1 exprimental support
for Windows NT4 and 2000 was added. The easiest way of Windows
installation is to download clickable Windows installer. It can be
downloaded from \url{http://klub.com.pl/dhcpv6/}. After downloading,
click on it and follow on screen instructions. Dibbler will be
installed and all required links will be placed in the Start
menu. Note that there are two Windows versions (ports): one for modern
systems (XP/2003/Vista and Win7) and one for archaic ones
(NT4/2000). Make sure to use proper port. If you haven't set up IPv6
support, see following sections for details.

Operation on Windows 8 was never tested, so support is not confirmed.

\subsection{Mac OS X installation}
As of 0.8.0 release, ready to use dmg packages are not provided,
therefore dibbler has to be compiled. Please follow section
\ref{compile} for generic Dibbler compilation that applies to Mac OS
X.

Currently support for Mac OS X is usable, but there is still one
notable limitation. Client is not able to configure DNS servers or
domain name informations.

\subsection{FreeBSD, NetBSD, OpenBSD, Solaris 11}
As of 0.8.1RC1 release, support for FreeBSD, NetBSD and OpenBSD has
been added. Solaris 11 support is implemented after 0.8.2 and will be included
in 0.8.3. There are no prebuilt binary packages available. Please
follow Section \ref{compile} for generic Dibbler compilation that applies to
all 3 mentioned OSes.


\subsection{Basic usage}
Depending what functionality do you want to use (server,client or relay),
you should edit configuration file (\verb+client.conf+ for client, \verb+server.conf+
for server and \verb+relay.conf+ for relay). All configuration files should
be placed in the \verb+/etc/dibbler+ directory. Also make sure that
\verb+/var/lib/dibbler+ directory is present and is writeable. After
editing configuration files, issue one of the following commands:

\begin{lstlisting}
dibbler-server start
dibbler-client start
dibbler-relay start
\end{lstlisting}

\verb+start+ parameter requires little explanation. It
instructs Dibbler to run in daemon mode -- detach from console and run
in the background. During configuration files fine-tuning, it is ofter better
to watch Dibbler's bahavior instantly. In this case, use \verb+run+
instead of \verb+start+ parameter. Dibbler will present its messages on
your console instead of log files. To finish it, press ctrl-c.

To stop server, client or relay running in daemon mode, type:
\begin{lstlisting}
dibbler-server stop
dibbler-client stop
dibbler-relay stop
\end{lstlisting}

To see, if client, server or relay are running, type:

\begin{lstlisting}
dibbler-server status
dibbler-client status
dibbler-relay status
\end{lstlisting}

To see full list of available commands, type \verb+dibbler-server+,
\verb+dibbler-client+ or \verb+dibbler-relay+ without any parameters.

If your OS uses different layout of directories, you may want to
modify Misc/Portable.h before starting compilation process.

\newpage
\section{Compilation}
\label{compile}
Dibbler is distributed in 2 versions: binary and source code. If there
is binary version provided for your system,  it is usually better
choice.  Compilation usually is performed by more experienced users.
In average case it does not offer significant advantages over binary version.
You probably want to just install and use Dibbler. If that is your case, read section
named \ref{install}. However, if you are skilled enough, you might
want to tune several Dibbler aspects during compilation. See \emph{
Dibbler Developer's Guide} for information about various compilation
parameters.

\subsection{Linux/Mac OS X/FreeBSD/NetBSD/OpenBSD/Solaris Compilation}
The following descriptions applies to Linux, Mac OS X, FreeBSD, NetBSD
and OpenBSD. Solaris 11 support has been added since 0.8.3.
Other POSIX systems may work, but were never tested by
author. If you would like to install Dibbler from sources, you will need all
required dependencies. In particular, you need a typical C++
environment: a C and C++ compilers (most probably gcc and g++), make,
and several other smaller tools.

To install Dibbler package from sources, go to project homepage and
download latest tar.gz source archive. Extract it using available tool
for that purpose (in most cases that would be tool called tar and
gzip).

After sources are extracted, they must be configured to match specific
operating system. To complete this step, a configure script must be
called:
\begin{lstlisting}
./configure
\end{lstlisting}

Configure script accepts many parameters, so if like to tweak
something, here is your chance. You may run \verb+./configure --help+
to see list of available parameters. For example, to set up sources to
compile in debug mode (useful if you want to debug them or provide
better bugreport), you can do this:

\begin{lstlisting}
./configure --enable-debug
\end{lstlisting}

Once configure completes its operation, it prints out details of its
configuration and source are ready for compilation. To build all
components, just type make. If you want to make specific component
only, you may use it as parameter to make, e.g.
\verb+make server+. After successful compilation type \verb+make install+ to
install compiled code in your system.

For example, to build server and relay, type:

\begin{lstlisting}
tar zxvf dibbler-0.8.1RC1-src.tar.gz
./configure
make server relay
make install
mkdir -p /var/lib/dibbler
\end{lstlisting}

Configure script was added in 0.8.1RC1. Earlier versions do not not
need that step.

Dibbler was compiled using gcc 2.95, 3.0, 3.2, 3.3, 3.4, 4.0, 4.1,
4.2, 4.3, 4.4, 4.5, 4.6 and 4.7 versions. Note that many older compilers are
now considered obsolete and were not tested for some time. Lexer files
(grammar defined config file) were generated using flex 2.5.35. Parser
file were created using bison++ 1.21.9. Flex and bison++ tools are not
required to compile Dibbler. Generated files are placed in GIT and in
tar.gz archives. Dibbler requires also make. Autoconf and automake
tools (autotools) were used for regeneration of the Makefiles and
configure script, but those generated files are shipped with the code,
so autotools should not be required.

\subsection{Modern Windows (XP...Win7) compilation}
Download dibbler-\version-src.tar.gz and extract it. In
\verb+Port-win32+ there are several project files (for server, client
and relay) for MS Visual Studio 2008. According to authors knowledge,
it is possible to compile dibbler using free MS Visutal C++ Express
2008 edition. Previous dibbler releases were compiled using MS Visual
Studio .NET (sometimes called 2002) and 2003. Those versions are not
supported anymore. It might work with newest dibber version, but there
are no guarantee. Open \verb+dibbler-win32.vs2008.sln+ solution file
click Build command. That should start compilation. After a while,
binary exe files will be stored in the \verb+Debug/+ or
\verb+Release/+ directories.

\subsection{Legacy Windows (NT/2000) compilation}
Windows NT4/2000 port is considered experimental, but there are reports
that it works just fine. To compile it, you should download dev-cpp
(\url{http://www.bloodshed.net/dev/devcpp.html}), a free IDE for
Windows utilising minGW port of the gcc for Windows. Run dev-cpp,
click ,,open project...'', and open one of the \verb+*.dev+ files located
in the Port-winnt2k directory, then click compile. You also should
take a look at \verb+Port-winnt2k/INFO+ file for details.


\subsection{IPv6 support}
Some systems does not have IPv6 enabled by default. In that is the case,
you can skip following subsections safely. If you are not sure, here is
an easy way to check it. To verify if you have IPv6 support, execute
following command: \verb+ping6 ::1+ (Linux) or \verb+ping ::1+
(Windows). If you get replies, you have IPv6 already installed.

\subsubsection{Setting up IPv6 in Linux}
Almost all modern Linux distributions have IPv6 enabled by default, so there
is very good chance that nothing has to be done. However, if that is
not the case, IPv6 can be enabled in Linux systems in two ways:
compiled directly into kernel or as a module. If you don't have IPv6
enabled, try to load IPv6 module: \verb+modprobe ipv6+ (command
executed as root) and try \verb+ping6 ::1+. If that fails, you have to
recompile kernel to support IPv6. There are numerous descriptions how
to recompile kernel available on the web, just type "kernel
compilation howto" in \href{http://www.google.com}{Google}.

\subsubsection{Setting up IPv6 in Windows Vista and Win7}
Both systems have IPv6 enabled by default. Also note that Win7 also
has DHCPv6 client built-in, so you may use it as well.

\subsubsection{Setting up IPv6 in Windows XP and 2003}
If you have already working IPv6 support, you can safely skip this section.
The easiest way to enable IPv6 support is to right click on the
\verb+My network place+ on the desktop, select \verb+Properties+, then locate
your network interface, right click it and select \verb+Properties+. Then
click \verb+Install...+, choose protocol and then IPv6 (its naming is
somewhat diffrent depending on what Service Pack you have installed).
In XP, there's much quicker way to install IPv6. Simply run command
\verb+ipv6 install+ (i.e. hit Start..., choose run... and then type
\verb+ipv6 install+). Also make sure that you have built-in firewall
disabled. See \emph{Frequently Asked Question} section for details.

\subsubsection{Setting up IPv6 in Windows 2000}
If you have already working IPv6 support, you can safely skip this
section. The following description was provided by Sob (
(\href{mailto:sob(at)hisoftware.cz}{sob(at)hisoftware.cz}). Thanks. This
description assumes that ServicePack 4 is already installed.

\begin{enumerate}
  \item Download the file tpipv6-001205.exe from:
    \url{http://msdn.microsoft.com/downloads/sdks/platform/tpipv6.asp}
    and save it to a local folder (for example, \verb+C:\IPv6TP+).
  \item From the local folder (\verb+C:\IPv6TP+), run
    \verb+Tpipv6-001205.exe+ and extract the files to the same
    location.
  \item From the local folder (\verb+C:\IPv6TP+), run
    \verb+Setup.exe -x+ and extract the files to a subfolder of the
    current folder (for example, \verb+C:\IPv6TP\files+).
  \item From the folder containing the extracted files
    (\verb+C:\IPv6TP\files+), open the file \verb+Hotfix.inf+ in a
    text editor.
  \item In the [Version] section of the Hotfix.inf file, change the
    line NTServicePackVersion=256 to NTServicePackVersion=1024, and
    then save changes. \footnote{This defines Service Pack
      requirement.  NTServicePackVersion is a ServicePack version
      multiplied by 256. If there would be SP5 available, this value
      should have been changed to the 1280.}
  \item From the folder containing the extracted files
    (\verb+C:\IPv6TP\files+), run \verb+Hotfix.exe+.
  \item Restart the computer when prompted.
  \item After the computer is restarted, from the Windows 2000
    desktop, click Start, point to Settings, and then click Network
    and Dial-up Connections. As an alternative, you can right-click My
    Network Places, and then click Properties.
  \item Right-click the Ethernet-based network interface to which you
    want to add the IPv6 protocol, and then click
    Properties. Typically, this network interface is named Local Area
    Connection.
  \item Click Install.
  \item In the Select Network Component Type dialog box, click
    Protocol, and then click Add.
  \item In the Select Network Protocol dialog box, click Microsoft
    IPv6 Protocol and then click OK.
  \item Click Close to close the Local Area Connection Properties
    dialog box.
\end{enumerate}

\subsubsection{Setting up IPv6 in Windows NT4}
If you have already working IPv6 support, you can safely skip this
section.  The following description was provided by The following
description was provided by Sob
(\href{mailto:sob(at)hisoftware.cz}{sob(at)hisoftware.cz}). Thanks.

\begin{enumerate}
  \item Download the file msripv6-bin-1-4.exe from:
    \url{http://research.microsoft.com/msripv6/msripv6.htm}{Microsoft}
    and save it to a local folder (for example, \verb+C:\IPv6Kit+).
  \item From the local folder (\verb+C:\IPv6Kit+), run
    \verb+msripv6-bin-1-4.exe+ and extract the files to the same
    location.
  \item Start the Control Panel's "Network" applet (an alternative way to do this is
    to right-click on "Network Neighborhood" and select "Properties") and select
    the "Protocols" tab.
  \item Click the "Add..." button and then "Have Disk...". When it asks you for
    a disk, give it the full pathname to where you downloaded the binary
    distribution kit (\verb+C:\IPv6Kit+).
  \item IPv6 is now installed.
\end{enumerate}
