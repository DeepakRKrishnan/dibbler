%%
%% Dibbler - a portable DHCPv6
%%
%% authors: Tomasz Mrugalski <thomson@klub.com.pl>
%%          Michal Kowalczuk <michal@kowalczuk.eu>
%%
%% released under GNU GPL v2 licence
%%

\newpage
\section{Server configuration}
\label{server-conf}
Server configuration is stored in \verb+server.conf+ file in the
\verb+/etc/dibbler+ (Linux systems) or in current (Windows systems)
directory.

\subsection{Scopes}

Configuration file can be logically split into separate ``sections''
that are called \emph{scopes}, for example interface scope contains
parameters related to configuration served over a given interface.
Some scopes can contain other scopes. Some commands are specific to a
given a given scope.

\subsubsection{Global scope}
\label{server-global-scope}
Every option can be declared in a global scope. Global options can be
defined here. Also options of a smaller scopes can be defined here --
they will be used as a default values. Configuration file has following syntax:

\begin{lstlisting}
 global-options
 interface-options
 class-options
 interface-declaration
\end{lstlisting}

\subsubsection{Interface declaration}
\label{server-iface-scope}
Each network interface, which should be serviced by the server, must be
mentioned in the configuration file. Network interface is defined like this:
\begin{lstlisting}
iface interface-name
{
  interface-options
  class-options
}
\end{lstlisting}

or

\begin{lstlisting}
iface number
{
  interface-options
  class-options
}
\end{lstlisting}

where \verb+interface-name+ denotes name of the interface and
\verb+interface-number+ denotes its number. Name no longer needs to be
enclosed in single or double quotes (except Windows systems, when
interface name contains spaces). Note that virtual interfaces, used
to setup relay support are also declared in this way.

\subsubsection{Address class scope}
\label{server-class-scope}
Class is a smallest scope used in the server configuration file. It
contains definition of the addresses, which will be provided to
clients. Only class scoped parameters can be defined here. Address class
is declared as follows:
\begin{lstlisting}
class
{
  class-options
  address-pool
}
\end{lstlisting}

Address pool defines range of the addresses, which can be assigned to the
clients. It can be defined in one of the following formats:
\begin{lstlisting}
pool minaddress-maxaddress
pool address/prefix
\end{lstlisting}

\subsubsection{Prefix class scope}
\label{server-pd-class-scope}
That is an equivalent of address class for a prefix delegation. It
contains definition of prefixes that are going to be delegation to
clients. Only pd-class scoped parameters can be defined here. Prefix
class is declared as follows:
\begin{lstlisting}
pd-class
{
    pd-pool prefix/length
    pd-length prefix-length
}
\end{lstlisting}

\subsubsection{Temporary address class scope}
\label{server-ta-class-scope}
That is an equivalent of address class for temporary addresses. It
contains definition of temporary addresses that are going to be
assigned to clients that request temporary addresses. Only ta-class
scoped parameters can be defined here. Prefix class is declared as
follows:
\begin{lstlisting}
ta-class {
    pool 2001:db8:1::1-2001:db81:1::ffff
}
\end{lstlisting}

\subsubsection{Routing scope}
\label{server-route-scope}
Support for routing configuration was added in 0.8.0RC1. It is
possible to define routing scope. Each scope represents a single
router available on-link. In this scope, routes available via
specified link my be defined.

\begin{lstlisting}
next-hop address-of-a-router
{
  route1-parameters
  route2-parameters
  ...
}
\end{lstlisting}

\subsubsection{Client scope}
\label{server-scope-client}
Server allows defining custom parameters on a per-host basis. See
Sections \ref{feature-exceptions} and \ref{example-server-exceptions}
for details. There are two types of reservations: DUID-based and
remote-id based. Following syntax can be used:

\begin{lstlisting}
client duid 00:00:00:00:00
{
    [address 2001:db8:1::]
    [prefix 2001:db8:1::/64]
    option1
    option2
    ...
}
\end{lstlisting}

\begin{lstlisting}
client remote-id 5-0x01020304
{
    [address 2001:db8:1::]
    [prefix 2001:db8:1::/64]
    option1
    option2
    ...
}
\end{lstlisting}

\subsubsection{Key scope}
Dibbler 0.8.3 introduced support for secure DNS Updates using TSIG
mechanism. Since this key is expected to be also used in DNS server
software, syntax is kept very similar to syntax accepted in
\href{http://www.isc.org/software/bind}{ISC BIND9} software. Note
semicolons at the end of each statement.

\begin{lstlisting}
key key-name {
   secret "base64encodedSecretHere==";
   algorithm algorithm-type;
   ...
};
\end{lstlisting}

\subsection{Server options}

So called standard options are defined by the base DHCPv6 specification,
a so called RFC 3315 document \cite{rfc3315}. Those options are
called standard, because all DHCPv6 implementations, should properly
handle them. Each option has a specific scope it belongs to.

Standard options are declared in the following way:

\begin{lstlisting}
OptionName option-value
\end{lstlisting}

\begin{description}
\item[work-dir] -- (scope: global). Takes one parameter of string
  type. Defines working directory.

\item[log-level] -- (scope: global). Takes one integer
  parameter. Defines verbose level of the log messages. The valid range
  is from 1 (very quiet) to 8 (very verbose). Those values are modelled
  after levels used in syslog. These are: 1(Emergency), 2(Alert),
  3(Critical), 4(Error), 5 (Warning), 6(Notice), 7(Info) and
  8(Debug). Currently Dibbler is using levels 3 to 8, as 1 and 2 are
  reserved for system wide emergency events.

\item[log-name] -- (scope: global). Takes one string
  parameter. Defines than name, which will be used during logging.

\item[log-mode] -- (scope: global). Takes one parameter that can be
  short, full, precise or syslog. Defines logging mode. In the
  default, full mode, name, date and time in the h:m:s format will be
  printed. In short mode, only minutes and seconds will be printed
  (this mode is useful on terminals with limited width). Precise mode
  logs information with seconds and microsecond precision. It is a
  useful as a performance diagnostic tool for finding bottlenecks in
  the DHCPv6 autoconfiguration process. Syslog works under POSIX
  systems (Linux, Mac OS X, BSD family) and allows default POSIX
  logging functions.

\item[log-colors] -- (scope: global). Takes one boolean parameter.
  Defines if logs printed to console should use colors. That feature
  is used to enhance logs readability.  As it makes the log files
  messy on systems that do not support colors, it is disabled by
  default. The default is off.

 \item[cache-size] -- (scope: global). Takes one parameter that
  specifies cache size in bytes. The default value is 1048576
  (1MB). It defines a size of the memory (specified in bytes) which
  can se used to store cached entries.

\item[stateless] -- (scope: global). It may be present or missing. The
  default is missing. Defines that server should run in stateless
  mode. In this mode only configuration parameters are defined, not
  addresses or prefixes. It is mutually exclusive
  with \emph{class}, \emph{ta-class} and \emph{pd-class}. See
  Section \ref{feature-stateless-stateful}.


\item[interface-id-order] -- (scope: global). Take one parameter that
  can be one of \verb+before+, \verb+after+ or \verb+omit+. The
        default is \verb+before+. This parameter defines placement of
        the interface-id option. During message relaying options can
        be placed in the \msg{RELAY-REPL} message is arbitrary
        order. This option has been specified to control that
        order. \opt{interface-id} option can be placed before or
        after \opt{relay-message} option. There is also possibility to
        instruct server to omit the \opt{interface-id} option
        altogether, but since this violates \cite{rfc3315}, it should
        not be used. In general, this configuration parameter is only
        useful when dealing with buggy relays, which can't handle all
        option orders properly. Consider this parameter a debugging
        feature. Note: similar parameter is available in the
        dibbler-relay.

\item[experimental] -- (scope: global). Allows enabling experimental
features. There are some highly-experimental features present in
Dibbler. To make a clear statement about their experimental nature,
user is required to acknowledge that fact by putting this statement in
its config file. This statement may be present or absent. The default
is absent.

\item[inactive-mode] -- (scope: global, type: present or missing,
  default: missing). This enables so called inactive mode. When server
  begins operation and it detects that required interfaces are not
  ready, error message is printed and server exits. However, if
  inactive mode is enabled, server sleeps instead and wait for
  required interfaces to become operational. That is a useful feature,
  when using wireless interfaces, which take some time to initialize
  as associate.

\item[accept-leasequery] -- (scope: interface). Takes one boolean
  parameter that specifies if server should support leasequery
  \cite{rfc5007} protocol on a given interface. The default value is
  0 (leasequery is not supported by default). See Section
  \ref{feature-leasequery}.

%%% TODO bulk-leasequery-accept
%%% TODO bulk-leasequery-tcp-port
%%% TODO bulk-leasequery-max-conns
%%% TODO bulk-leasequery-timeout

\item[guess-mode] -- (scope: global, type: present or missing,
  default: missing). When this option is enabled, server will not pay
  close attention to the interface-id option in relayed messages. If
  interface-id has a value other than specified in server.conf or even
  when there is no interface-id option at all, it will use first relay
  defined.

\item[script] -- (scope: global). Takes one string parameter that
  specifies name of a script that will be called every time something
  important happens in a system, e.g. when address or prefix is
  assigned, updated or released. See Section \ref{feature-script}.

\item[fqdn-ddns-address] -- (scope: global). Takes one parameter that
  specifies address of DNS server that will be used for DNS
  Updates. See Section \ref{feature-dns-update}.

\item[ddns-protocol] -- (scope: global). Takes one string
parameter. Defines protocol that should be used during DNS Update
mechanism. Allowed values are \verb+tcp+, \verb+udp+ and \verb+any+.
Any means that UDP will be tried first and if it fails, update will be
retried over TCP. See Section \ref{feature-dns-update}.

\item[ddns-timeout] -- (scope: global). Takes one integer parameter
that specifies timeout in milliseconds. Defines how long client should
wait for DNS server response during DNS Update before declaring
update a failure. See Section \ref{feature-dns-update}.

\item[class] -- (scope: interface). This definition must be followed by
curly braces and creates a new address class scope. See
Section \ref{server-class-scope}.

\item[pd-class] -- (scope: interface). This definition must be
followed by curly braces and creates a new prefix-delegation class
scope. See Section \ref{server-pd-class-scope}.

\item[ta-class] -- (scope: interface). This definition must be
followed by curly braces and creates a new temporary address class
scope. See Section \ref{server-ta-class-scope}.

\item[next-hop] -- (scope: interface). This definition takes one
parameter that defines IPv6 address of a router. Without any further
parameters, it conveys an information about default route for
bandwidth limited networks. That mode is discouraged, unless there are
significant bandwith limitations. It is usually followed by curly
braces that create a new route scope. See Section \ref{server-route-scope}.

 \item[preference] -- (scope: interface, type: 0-255, default:
            none). Eech server can be configured to a specific
            preference level. When client receives several
            \msg{ADVERTISE} messages, it should choose that server,
            which has the highest preference level. It is also worth
            noting that client, upon reception of the \msg{ADVERTISE}
            message with preference set to 255 should skip wait phase
            for possible other \msg{ADVERTISE} messages.


 \item[unicast] -- (scope: interface, type: address,
            default:none). Normally clients sends data to a well known
            multicast address. This is easy to achieve, but it wastes
            network resources as all nodes in the network must process
            such messages and also network load is increased. To prevent
            this, server might be configured to inform clients about its
            unicast address, so clients, which accept it, will switch to
            a unicast communication.

 \item[rapid-commit] -- (scope: interface, type: boolean, default:
            0). This option allows rapid commit procedure to be
            performed. Note that enabling rapid commit on the server
            side is not enough. Client must be configured to allow
            rapid commit, too.

\item[iface-max-lease] -- (scope: interface, type: integer, default:
            $2^{32}-1$). This parameter defines, how many normal
            addresses can be granted on this interface.

\item[client-max-lease] -- (scope: interface, type: interger,
            default:$2^{32}-1$). This parameter defines, how many
            addresses one client can get. Main purpose of this
            parameter is to limit number of used addresses by
            misbehaving (malicious or restarting) clients.

\item[relay] -- (scope: interface). Takes one string or integer
  parameter that designated interface name or interface index. It is
  used in relay definition.  It specifies name of the physical (or
  name of another relay, if cascade relaying is used) interface, which
  is used to receive and transmit relayed data. See
  \ref{feature-relays} for details of relay deployment and sections
  \ref{example-server-relay1} and \ref{example-server-relay2} for
  configuration examples.

\item[interface-id] -- (scope: interface, type: integer, default: not
  defined). Used in relay definition. Each relay interface should have
  defined its unique identified. It will be sent in the
  \opt{interface-id} option. Note that this value must be the same as
  configured in the dibbler-relay. It may be possible to specify this
  parameter by using a number (option will be 4 bytes long), a string
  or a hex of arbitrary length (please use the same format as for
  DUID). See \ref{feature-relays}, \ref{example-server-relay1} and
  \ref{example-relay} for details.

 \item[vendor-spec] -- (scope: interface, type: integer-hexstring,
   default: not defined). This parameter can be used to configure some
   vendor-specific information option. Since there are no
   dibbler-specific options, this implementation is flexible. User can
   specify in the configuration file, how should this option look
   like. See \ref{example-server-vendor-spec} section for details. It
   is uncommon, but possible to define several vendor specific options
   for different vendors. In such case, administrator must specify
   coma separated list. Each list entry is a vendor (enterprise
   number), ,,--'' sign and a hex dump (similar to DUID).

 \item[pool] -- (scope: class). Takes coma separated IPv6 address
   ranges. Each range is defined as first-address, a dash and a second
   address. Defines a range of available addresses that will be
   assigned in specific class. An example pool definition looks like
   this:
   \begin{lstlisting}
     pool 2001:db8:abcd:: - 2001:db8:abcd::ffff
   \end{lstlisting}
   It is also possible to use prefix/length notation.

 \item[pd-pool] -- (scope: pd-class). Takes coma separated IPv6
   address ranges. Each range is defined as fist-address, a dash and a
   second address. Defines a range of available prefixes (only
   prefixes themselves, not their lengths) that will be assigned in
   specific class. An example pd-pool definition looks like this:
   \begin{lstlisting}
     pd-pool 2001:db8:abcd:: - 2001:db8:abcd::ffff
   \end{lstlisting}
   It is also possible to use prefix/length notation.

\item[share] -- (scope: class). Defines percentage of clients that a
  class should handle. This parameter is only useful if there are more
  then one class defined. See Section
  \ref{example-server-multiple-classes}.

 \item[T1] -- (scope: class, type: integer or integer range: default:
   $2^{32}-1$). This value defines after what time client should start
   renew process. Exact value or accepted range can be specified. When
   exact value is defined, client's hints are ignored completely.

 \item[T2] -- (scope: class, type: integer or integer range,
   default:$2^{32}-1$). This value defines after what time client will
   start emergency rebind procedure if renew process fails. Exact
   value or accepted range can be specified. When exact value is
   defined, client's hints are ignored completely.

\item[valid-lifetime] (scope: class, type: integer or integer range,
            default:$2^{32}-1$). This parameter defines valid lifetime of
            the granted addresses. If range is specified, client's
            hints from that range are accepted.

\item[preferred-lifetime] (scope: class, type: integer or integer range,
            default:$2^{32}-1$). This parameter defines prefered
            lifetime of the granted addresses. If range is specified,
            client's hits from that range will be accepted.

\item[class-max-lease]  -- (scope: interface, type: interger,
            default:$2^{32}-1$). This parameter defines, how many
            addresses can be assigned from that class.

\item[reject-clients] -- (scope: class, type: address or DUID list,
            default: none). This parameter is sometimes called
            black-list. It is a list of a clients, which should not be
            supported. Clients can be identified by theirs link-local
            addresses or DUIDs.

\item[accept-only] -- (scope: class, type: address or DUID list,
            default: none). This parameter is sometimes called
            white-list. It is a list of supported clients. When this
            list is not defined, by default all clients (except
            mentioned in reject-clients) are supported. When
            accept-only list is defined, only client from that list
            will be supported.

\item[addr-params] -- (scope: class). Experimental feature that takes
  one boolean parameter. It defines prefix length that is configured
  in addr-params option. See Section \ref{feature-addr-params}.

\item[allow] -- (scope: class). Specifies that clients that belong to
  a specific client class are allowed to use that address class. Takes
  one string parameter that defines client class name. See Section
  \ref{feature-client-class}.

\item[deny] -- (scope: class). Specifies that clients that belong to
  a specific client class are denied use of that address class. Takes
  one string parameter that defines client class name. See Section
  \ref{feature-client-class}.

 \item[option dns-server] -- (scope: interface, type: address list, default:
   none). This option conveys information about DNS servers
   available. After retriving this information, clients will be able
   to resolve domain names into IP (both IPv4 and IPv6)
   addresses. Defined in \cite{rfc3596}.

 \item[option domain] -- (scope: interface, type: domain list, default:
   none). This option is used for configuring one or more domain
   names, which clients are connected in. For example, if client's
   hostname is \verb+alice.mylab.example.com+ and it wants to contact
   \verb+bob.mylab.example.com+, it can simply refer to it as
   \verb+bob+. Without domain name configured, it would have to use
   full domain name. Defined in \cite{rfc3596}.

 \item[option ntp-server] -- (scope: interface, type: address list, default:
   none). This option defines information about available NTP
   servers. Network Time Protocol \cite{rfc2030} is a protocol used
   for time synchronisation, so all hosts in the network has the same
   proper time set. Defined in \cite{rfc4075}.

 \item[option time-zone] -- (scope: interface, type: timezone, default:
   none). It is possible to configure timezone, which is provided by
   the server. Note that this option is considered obsolete as it is
   mentioned in draft version only \cite{draft-timezone}. Work on this
   draft seems to be abandoned as similar functionality is provided by
   now standard \cite{rfc4075}.

 \item[option sip-server] -- (scope: interface, type: address list, default:
   none). Session Initiation Protocol \cite{rfc3263} is an control
   protocol for creating, modifying, and terminating sessions with one
   or more participants. These sessions include Internet telephone
   calls, multimedia distribution, and multimedia conferences. Its
   most common usage is VoIP. Format of this option is defined in
   \cite{rfc3319}.

 \item[option sip-domain] -- (scope: interface, type: domain list, default:
   none). It is possible to define domain names for Session Initiation
   Protocol \cite{rfc3263}. Configuration of this parameter will ease
   usage of domain names in the SIP protocol. Format of this option is
   defined in \cite{rfc3319}.

 \item[option nis-server] -- (scope: interface, type: address list, default:
   none). Network Information Service (NIS) is a Unix-based system
   designed to use common login and user information on multiple
   systems, e.g. universities, where students can log on to ther
   accounts from any host. Its format is defined in \cite{rfc3898}.

 \item[option nis-domain] -- (scope: interface, type: domain list, default:
   none). Network Information Service (NIS) can albo specify domain
   names. It can be configured with this option. It is defined in
   \cite{rfc3898}.

 \item[option nis+-server] -- (scope: interface, type: address list, default:
   none). Network Information Service Plus (NIS+) is an improved
   version of the NIS protocol. This option is defined in
   \cite{rfc3898}.

 \item[option nis+-domain] -- (scope: interface, type: domain list, default:
   none). Similar to nis-domain, it defines domains for NIS+. This
   option is defined in \cite{rfc3898}.

 \item[option lifetime] -- (scope: interface, type: boolean, default:
   no). Base spec of the DHCPv6 protocol does offers way of refreshing
   addresses only, but not the options. Lifetime defines, how often
   client should renew all its options. When defined, lifetime option
   will be appended to all replies, which server sends to a client. If
   client does not support it, it should ignore this option. Format of
   this option is defined in \cite{rfc4242}.

 \item[option fqdn] -- (scope: interface). Takes 0, 1 or 2 integer
   parameters that are followed by FQDN list. Additional integer
   parameters designate fqdn-mode and reverse zone length in DNS
   Update. FQDN-mode can have 3 values: 2 (both AAAA and PTR record
   will be updated by server), 1 (server will update PTR only) or
   0 (server will not update anything). Reverse zone length is an
   integer between 0 and 128 and designates reverse zone length. FQDN
   list is a coma separated list of fully qualified domain names, with
   possible reservations for DUIDs or addresses. FQDN mechanism is
   defined in \cite{rfc4704}. See Section \ref{feature-dns-update}.

\item[accept-unknown-fqdn] -- (scope: Interface). Takes one integer
  parameter, possibly followed by second string parameter that
  designated domain name. It specifies how server should react to
  incoming FQDN options that contain names that are unknown to the
  server. Allowed values are 0 (reject), 1 (send other name from
  allowed list), 2 (accept any name client sends), 3(accept any name
  client sends, but append specified domain suffix) and 4 (ignore
  client's hint, generate name based on his address, append domain
  name). Choices 3 and 4 require additional string parameter that
  defines domain suffix. See Sections \ref{feature-dns-update}
  and \label{example-server-fqdn}.

\item[option] -- (scope: interface). Takes one integer number followed
  by several possible parameter combinations. It defines custom
  option that server may send out to clients. Supported formats are:
  \begin{lstlisting}
    option number - DUID
    option number address-keyword address
    option number address-list
    option number string-keyword string
  \end{lstlisting}
Where number is an integer that defined option type, DUID is a
hex-formatted string that defines option content, address-keyword is a
word ``address'', address is an IPv6 address, address-list is coma
separated list of addresses, string-keyword is a word ``string'' and
string is any string enclosed in single or double quotes. See Section
\ref{feature-custom-options} and \ref{example-server-custom}.

 \item[option aftr] -- (scope: interface, type: FQDN). In Dual-Stack Lite
   networks, client may want to configure DS-Lite tunnel. Client may
   want to obtain information about AFTR (a remote tunnel
   endpoint). This option conveys a fully qualified domain name of the
   remote tunnel. This option is defined in \cite{rfc6334}.

 \item[option neighbors] -- (scope: interface). Experimental feature
   for Remote Autoconfiguration. Do not use it unless you know exactly
   what you are doing. Takes coma separated list of addresses. This
   option requires \emph{experimental} mode to be enabled. See Section
   \ref{feature-remote-autoconf}.

 \item[auth-method] -- (scope: global, type: string, default:
   empty). Set it to one of the following values to enable
   authentication on ther server side, using selected method of
   generating authentication information:
   \texttt{none}, \texttt{digest\_plain}, \texttt{digest\_hmac\_md5},
   \texttt{digest\_hmac\_sha1}, \texttt{digest\_hmac\_sha224},
   \texttt{digest\_hmac\_sha256}, \texttt{digest\_hmac\_sha384},
   and \texttt{digest\_hmac\_sha512}.

 \item[auth-lifetime] -- (scope: global, type: integer, default:
   0). Authentication lifetime. Currently not supported.

 \item[auth-key-len] -- (scope: global, type: integes, default:
   32). Key generation nonce length (see \cite{draft-aaa} for
   details).

\item[client-class] -- (scope: global). Takes one string parameter
  that defines name of a client class. Client class name is followed
  by curly brackets that create client-class scope. Clients can be
  grouped into classes depending on rules defined in
  client-class. This can be used together with \verb+allow+ and
  \verb+deny+ to assign segregate clients into different groups. See
  Section \ref{feature-client-class} for overview and Section
  \ref{class-expressions} for list of supported expressions.

\item[address] -- (scope: client). Takes one parameter that specifies
   address. It instructs server to reserve this particular address for
   defined client. See Sections \ref{feature-exceptions}
   and \ref{example-server-exceptions} for details.

\item[prefix] -- (scope: client). Takes one parameter that specifies
   prefix using prefix/length notation. It instructs server to reserve
   specified prefix for defined client. See Sections \ref{feature-exceptions}
   and \ref{example-server-exceptions} for details.

\item[key] -- (scope: global). Take one string parameter that
   specifies key name. This keyword instructs server to create a key
   with a specified name. This key will be used for TSIG in DNS
   Updates. It is followed by curly braces that open up a new key
   scope. Closing curly brace is followed by a semicolon.

\item[secret] -- (scope: key). Takes one string parameter and is
   followed by a semicolon. That parameter is a base64-encode secret
   value of the key. It is mandatory if key scope is defined.
\item[algorithm] -- (scope: key). Takes one enum argument that is
   followed by a semicolon. This parameter is mandatory if the key
   scope is present. It specifies agorithm for the key. Currently the
   supported value is hmac-md5.
\item[fudge] -- (scope: key). Takes one integer parameter that is
   followed by a semicolon. Each TSIG signature is valid for a
   specified amount of time only. This optional parameter specifies
   period in which TSIG is valid, expressed in seconds. If missing,
   the default value of 300 is used. The allowed values are between 0
   and 65535.
\end{description}

\subsubsection{Client class quantifiers}
\label{class-expressions}
Additional parameters are used during client class definition. See
Section \ref{feature-client-class} for details and examples.

\begin{description}
\item[match-if] -- (scope: client-class).
\item[contain] -- (scope: client-class).
\item[substring] -- (scope: client-class).
\item[==] -- (scope: client-class).
\item[and] -- (scope: client-class).
\item[or] -- (scope: client-class).
\item[client.vendor-spec.en] -- (scope: client-class).
\item[client.vendor-spec.data] -- (scope: client-class).
\item[client.vendor-class.en] -- (scope: client-class).
\item[client.vendor-class.data] -- (scope: client-class).
\end{description}


\subsection{Server configuration examples}

This subsection contains various examples of the server
configuration. If you are interested in additional examples, download
source version and look at \verb+*.conf+ files.

\subsubsection{Example 1: Simple}

In opposite to client, server uses only interfaces described in config
file. Let's examine this common situation: server has interface named
\emph{eth0} (which is fourth interface in the system) and is supposed
to assign addresses from 2000::100/124 class. Simplest config file
looks like that:

\begin{lstlisting}
# server.conf
iface eth0
{
  class
  {
    pool 2000::100-2000::10f
  }
}
\end{lstlisting}

\subsubsection{Example 2: Timeouts}
Server should be configured to deliver specific timer values to the
clients. This example shows how to instruct client to renew (T1 timer)
addresses one in 10 minutes. In case of problems, ask other servers in
15 minutes (T2 timer), that allowe prefered lifetime range is from 30
minutes to 2 hours, and valid lifetime is from 1 hour to 1 day. DNS
server parameter is also provided. Lifetime option is used to make
clients renew all non-address related options renew once in 2 hours.

\begin{lstlisting}
# server.conf
iface eth0
{
  T1 600
  T2 900
  prefered-lifetime 1800-3600
  valid-lifetime 3600-86400
  class
  {
    pool 2000::100/80
  }

  option dns-server 2000::1234
  option lifetime 7200
}
\end{lstlisting}

\subsubsection{Example 3: Limiting amount of addresses}
Another example: Server should support 2000::0/120 class on eth0
interface. It should not allow any client to obtain more than 5
addresses and should not grant more then 50 addresses in total. From
this specific class only 20 addresses can be assigned. Server
preference should be set to 7. This means that this server is more
important than all server with preference set to 6 or less.
Config file is presented below:

\begin{lstlisting}
# server.conf
iface eth0
{
  iface-max-lease 50
  client-max-lease 5
  preference 7
  class
  {
    class-max-lease 20
    pool 2000::1-2000::100
  }
}
\end{lstlisting}

\subsubsection{Example 4: Unicast communication}
\label{example-server-unicast}

Here's modified previous example. Instead of specified limits, unicast
communication should be supported and server should listen on
2000::1234 address. Note that default multicast address is still
supported. You must have this unicast address already configured on
server's interface.

\begin{lstlisting}
# server.conf
log-level 7
iface eth0
{
  unicast 2000::1234
  class
  {
    pool 2000::1-2000::100
  }
}
\end{lstlisting}

\subsubsection{Example 5: Rapid-commit}
This configuration can be called quick. Rapid-commit is a way to shorten exchange to only two messages. It is
quite useful in networks with heavy load. In case if client does not
support rapid-commit, another trick is used. Preference is set to
maximum possible value. 255 has a special meaning: it makes client to
skip wait phase for possible advertise messages from other servers and
quickly request addresses.

\begin{lstlisting}
# server.conf
log-level 7
iface eth0
{
  rapid-commit yes
  preference 255
  class
  {
    pool 2000::1/112
  }
}
\end{lstlisting}

\subsubsection{Example 6: Access control}
Administrators can selectively allow certain client to use this
server (white-list). On the other hand, some clients could be
explicitly forbidden to use this server (black-list). Specific DUIDs,
DUID ranges, link-local addresses or the whole address ranges are
supported. Here is config file:

\begin{lstlisting}
# server.conf
iface eth0
{
  class
  {
    # duid of the rejected client
    reject-clients ``00001231200adeaaa''
    2000::2f-2000::20  // it's in reverse order, but it works.
                       // just a trick.
  }
}
iface eth1
{
  class
  {
    accept-only fe80::200:39ff:fe4b:1abc
    pool 2000::fe00-2000::feff
  }
}
\end{lstlisting}

\subsubsection{Example 7: Multiple classes}
\label{example-server-multiple-classes}
Although this is not common, a few users have requested support for multiple classes on one interface.
Dibbler server can be configured to use several classes. When client asks for an address, one of the classes
is being choosen on a random basis. If not specified otherwise, all classes have equal probability of being chosen.
However, this behavior can be modified using \verb+share+ parameter. In the following example, server supports
3 classes with different preference level: class 1 has 100, class 2 has 200 and class 3 has 300. This means that class 1
gets $\frac{100}{100+200+300} \approx 16\% $ of all requests, class 2
gets $\frac{200}{100+200+300} \approx 33\% $ and class 3 gets the rest
($\frac{300}{100+200+300}=50\% $).

\begin{lstlisting}
# server.conf
log-level 7
log-mode short

iface eth0 {
 T1 1000
 T2 2000

 class {
   share 100
   pool 4000::1/80
 }
 class {
   share 200
   pool 2000::1-2000::ff
 }

 class {
   share 300
   pool 3000::1234:5678/112
 }
}
\end{lstlisting}

\subsubsection{Example 8: Relay support}
\label{example-server-relay1}
To get more informations about relay configuration, see section \ref{feature-relays}.
Following server configuration example explains how to use
relays. There is some remote relay with will send encapsulated data over
eth1 interface. It is configured to append interface-id option set to
5020 value. Let's allow all clients using this relay some addresses
and information about DNS servers. Also see section
\ref{example-relay-1} for corresponding relay configuration.

Note that although eth1 interface is mentioned in the configuration file,
direct traffic from clients located on the eth1 interface will not be
supported. In this example, eth1 is used only to support requests
relayed from remote link identified with interface-id value 5020.
Of course it is possible to support both local and remote traffic. In
such case, normal eth1 definition should be present in the server
configuration file. Also note that real (physical) interfaces should
be specified before logical ones.

\begin{lstlisting}
# server.conf
iface relay1 {
  relay eth1
  // interface-id 5020
  // interface-id "some interface name"
  interface-id 0x427531264361332f3000001018680f980000

  class {
    pool 2000::1-2000::ff
  }
  option dns-server 2000::100,2000::101
}
\end{lstlisting}

\subsubsection{Example 9: Cascade 2 relays}
\label{example-server-relay2}
This is an advanced configuration. It assumes that client sends data to
relay1, which encapsulates it and forwards it to relay2, which
eventually sends it to the server (after additional encapsulation). It
assumes that first relay adds interface-id option set to 6011 and
second one adds similar option set to 6021. For details about relays
in general and cascade setup in particular, see section
\ref{feature-relays}. Also see section \ref{example-relay-cascade}
for corresponding relays configuration.

\begin{lstlisting}
# server.conf
iface relay1
{
  relay eth0
  interface-id 6011
}

iface relay2
{
  relay relay1
  interface-id 6021
  T1 1000
  T2 2000
  class {
    pool 6020::20-6020::ff
  }
}
\end{lstlisting}

\subsubsection{Example 10: Dynamic DNS (FQDN)}
\label{example-server-fqdn}

Support for Dynamic DNS Updates was added in version 0.5.0RC1. To
configure it on the server side, list of available names usually
should be defined. Each name can be reserved for a certain address or
DUID. When no reservation is specified, it will available to everyone,
i.e. the first client asks for FQDN will get this name. In following
example, name 'zebuline.example.com' is reserved for address 2001::db8:1::1,
kael.example.com is reserved for 2001:db8:1::2 and test.example.com is
reserved for client using DUID
00:01:00:00:43:ce:25:b4:00:13:d4:02:4b:f5.

Also note that is required to define, which side can perform updates.
This is done using single number after ,,option fqdn'' phrase. Server
can perform two kinds of DNS Updates: AAAA (forward resolving,
i.e. name to address) and PTR (reverse resolving, i.e. address to
name). To configure server to execute both updates, specify 2. This is
a default behavior. If this value will be skipped, server will attempt
to perform both updates. When 1 will be specified, server will update
PTR record only and will leave updating AAAA record to the
client. When this value is set to 0, server will not perform any
updates.

The last parameter (64 in the following example) is a prefix length of
the reverse domain supported by the DNS server, i.e. if this is set to
64, and 2000::/64 addresses are used, DNS server must support
0.0.0.0.0.0.0.0.0.0.0.0.0.0.2.ip6.arpa. zone.

There are several additional parameters that affect DNS Update
mechanism. \verb+ddns-protocol+ specifies protocol that should be used
for communication with DNS server.  Allowed values
are \verb+udp+, \verb+tcp+ or \verb+any+. ``Any'' will try to use UDP
and if that fails, it will revert to TCP. Second parameter
is \verb+ddns-timeout+ that specifies maximum time allowed for DNS
server to respond before assuming communication failure. It is
specified in milliseconds.

The next useful parameter is \verb+fqdn-ddns-address+ that specifies
address of DNS server that updates should be performed to. If it is
not specified, first DNS address from \verb+option dns-server+ will be
used.

The last important parameter is +\verb+accept-unknown-fqdn+. In a
simplest scenario, server is configured with a list of allowed
names. Connecting clients may get only those names. That is convenient
case, but it is often not feasible to deploy it in a real network.  In
real networks, clients usually send out their own names, rather than
wait for server to assign them one. In such cases, it is somewhat
expected that server will not have complete list of all possible names
that clients may send. Thus sooner or later server will likely receive
a fqdn name that is unknown to him. This parameter specifies server
behavior in such case.

There are 5 possible policied currently supported. Each one is
identified with an integer between 0 and 4. 0 means that uknown names
should be rejected. That policy is useful for strictly controlled
networks. 1 means that other available name from list of possible
names should be sent instead. This is a compromise between strict
control over names and liberal acceptance of clients' names. Policy 2
accept any name that client will send. Names will be sanity
checked. Note that mobile and nomadic clients may send names from
their home networks. That may be a problem if server attempts to
update AAAA records as its DNS server will probably only accept AAAA
updates for locally administered domains. As a solution to this
problem, policy 3 was implemented. It takes client's hostname (without
its domain name), appends local domain name and uses such constructed
fully qualified domain name. For example, if client
sends \verb+nomad.faraway.org+ while visiting \verb+example.org+, with
this policy in place, \verb+nomad.example.org+ will be assigned.

The last policy is useful for larger networks. Instead of accepting
clients' ideas about their hostnames, dedicated name is generated
based on assigned address. For example, client that received
\verb+2001:db8:1:0:c7a8:e81:c500:46ce+ address in domain
\verb+example.org+ will be assigned
a \verb+2001-db8-1-0-c7a8-e81-c500-46ce.example.org+ name.

Following config file is a good starting point for tweaking DNS
Update enabled server configuration.

\begin{lstlisting}
# server.conf

# Logging level range: 1(Emergency)-8(Debug)
#
log-level 8

# Don't log full date
log-mode short

# Set protocol to one of the following values: udp, tcp, any
ddns-protocol udp

# Sets DDNS Update timeout (in ms)
ddns-timeout 1000

# specify address of DNS server to be used for DDNS
fqdn-ddns-address 2001::1

iface "eth0" {

# assign addresses from this class
 class {
   pool 2001:db8:1::/64
 }

# provide DNS server location to the clients
# also server will use this address to perform DNS Update,
# so it must be valid and DNS server must accept DNS Updates.
 option dns-server 2000::1

# provide their domain name
 option domain example.com

# provide fully qualified domain names for clients
# note that first, second and third entry is reserved
# for a specific address or a DUID
 option fqdn 1 64
             zebuline.example.com - 2000::1,
             kael.example.com - 2000::2,
             wash.example.com - 0x0001000043ce25b40013d4024bf5,
             zoe.example.com,
             malcolm.example.com,
             kaylee.example.com,
             jayne.example.com,
             inara.example.com

# specify what to do with client's names that are not on the list
# 0 - reject
# 1 - send other name from allowed list
# 2 - accept any name client sends
# 3 - accept any name client sends, but append specified domain suffix
# 4 - ignore client's hint, generate name based on his address, append domain name

 accept-unknown-fqdn 4 example.org

}
\end{lstlisting}

\subsubsection{Example 11: Vendor-specific Information option}
\label{example-server-vendor-spec}
It is possible to configure dibbler-server to provide vendor-specific
information options. Since there are no dibbler-specific parameters,
this implementation is quite flexible. Enterprise number as well as
content of the option itself can be configured.

\begin{lstlisting}
# server.conf
log-level 8
log-mode precise
iface "eth1" {
 class {
   pool 2000::1-2000::ff
 }

 option vendor-spec 1234-0x00002fedc
}
\end{lstlisting}

In some rare cases, several different options for different vendors
may be specifed. In the folloging example 2 different values are
defined, depending on which vendor client will specify in \msg{SOLICIT} or
\msg{REQUEST} message. If client will only mention that it is interested in
any vendor specific into (i.e. did not sent \opt{vendor-spec info} option, but
only mentioned in in \opt{option request} option, it will receive
first vendor option defined (in the following example, that would be a
1234 and 0002fedc).

\begin{lstlisting}
# server.conf
log-level 8
log-mode precise
iface "eth1" {
 class {
   pool 2000::1-2000::ff
 }

 option vendor-spec 1234-0x00002fedc,5678-0x0002aaaa
}
\end{lstlisting}

\subsubsection{Example 12: Per client configuration}
\label{example-server-exceptions}
Usually all clients receive the same configuration options, e.g. all
clients will use the same DNS server. However, it is possible to
specify that particular clients should receive different options than
others. Following example set DNS server to 2000::1, domain
to example.com and vendor specific information for vendor 5678.
However, if requesting client has DUID 00:01:02:03:04:05:06:07:08, it
will receive different parameters (second.client.biz domain,
1234::5678:abcd as a DNS server and finally different vendor-specific
information). Also client with DUID 0x0001000044e8ef3400085404a324
will receives normal domain and DNS server, but different (vendor=2)
vendor specific information. See section \ref{feature-exceptions} for
background information. Since 0.8.0RC1, also addresses can be reserverd
in this way.

Addresses reserved for special clients may be inside or outside of
specified pools. If leases are outside of specified pools, timers (t1,
t2, prefered and valid lifetimes are set to the default values). It is
currently not possible to specify separate timers (t1, t2, preferred
or valid lifetimes) on a per-client basis. If reservations are out of
pool, timers applicated to the interface will be used. See second
example in this section.

Note that per client reservation was significantly refactored after
0.8.2, so its stability is not yet confirmed.

\begin{lstlisting}
# server.conf
# Example server configuration file: per-client configuaration
#
# In this example, some clients receive different parameters than others.

# Logging level range: 1(Emergency)-8(Debug)
#
log-level 8

# Don't log full date
log-mode short

iface "eth0" {

    class {
        pool 2001:db8:1::/64
    }

    pd-class {
        pd-pool 2001:db8:2::/48
        pd-length 64
    }

  # common configuration options, provided for all clients
  option dns-server 2001:db8:1::1
  option domain example.com
  option vendor-spec 5678-2-0xaaaa,1234-3-0x0102

  # special parameters for client with DUID 00:01:02:03:04:06
  client duid 00:01:02:03:04:06
  {
        address 2001:db8:1::123
        prefix 2001:db8:abcd::/64
        option domain second.client.biz
        option dns-server 2001:db8::5678:abcd
        option vendor-spec 5678-2-0xbbbb, 1234-5-0x222222
  }

  # this client should receive default domain and dns-server,
  # but different vendor-spec info
  client duid 0x0001000044e8ef3400085404a324
  {
        option vendor-spec 1111-57-0x01020304
  }

  client remote-id 5-0x01020304
  {
        address 2001:db8:1::0102:0304
        option domain our.special.remoteid.client.org
  }

  #  client link-local fe80::1:2:3:4
  #  {
  #      option domain link.local.detected.interop.test.com
  #  }

}

\end{lstlisting}

The following example shows out of pool reservation. Regular clients
will get addresses from the 2001:db8:123::/64 pool. However, the
client with DUID 00:01:00:0a:0b:0c:0d:0e:0f will get an 2002::babe
address that does not belong to any configured pool. That particular
client with get parameters from the interface on which this
exception was defined. In this discussed example, what will be
t1=1000, t2=2000, preferred-lifetime=3000 and valid-lifetime=4000.

\begin{lstlisting}
iface eth0 {
    t1 1000
    t2 2000
    preferred-lifetime 3000
    valid-lifetime 4000
    class { pool 2001:db8:123::/64 }
    client duid 00:01:00:0a:0b:0c:0d:0e:0f {
        address 2002::babe
    }
}
\end{lstlisting}

\subsubsection{Example 13: Prefix delegation}
\label{example-server-prefix}

Prefix delegation works quite similar to normal address granting.
Administrator defines pool and server provides prefixes from that
pool. Before using prefix delegation, please read section
\ref{feature-prefix}. Client configuration example is described in section
\ref{example-client-prefix}.

\begin{lstlisting}
# server.conf
log-mode precise

iface "eth0" {

 # the following lines instruct server to grant each client
 # prefix for this pool. For example, client might get
 # 2222:2222:2222:2222:2222:993f::/96
 pd-class {
        pd-pool 2222:2222:2222:2222:2222::/80
        pd-length 96
        T1 11111
        T2 22222
    }

}
\end{lstlisting}

\subsubsection{Example 14: Multiple prefixes}
\label{example-server-prefixes}
It is possible to define more than one pool, so each client will
receive several prefixes. It is necessary to define each pool with the
same length, i.e. it is not possible to mix different pool lengths.
See section \ref{feature-prefix} for prefix delegation background
information. Client configuration example is described in section
\ref{example-client-prefix}.

\begin{lstlisting}
# server.conf
log-mode precise

iface "eth0" {

 T1 1800
 T2 2700
 prefered-lifetime 3600
 valid-lifetime 7200

 # provide addresses from this pool
 class {
   pool 5000::/48
 }

 # the following lines instruct server to grant each client
 # 2 prefixes. For example, client might get
 # 2222:2222:2222:2222:2222:993f:6485::/96 and
 # 1111:1111:1111:1111:1111:993f:6485::/96
 pd-class {
        pd-pool 2222:2222:2222:2222:2222::/80
        pd-pool 1111:1111:1111:1111:1111::/80
        pd-length 96
        T1 11111
        T2 22222
    }

}
\end{lstlisting}

\subsubsection{Example 15: Inactive mode}
\label{example-server-inactivemode}
See sections \ref{example-client-inactivemode} and
\ref{feature-inactive-mode} for inactive mode explanation.
The same behavior has been added for server.

\begin{lstlisting}
#server.conf

log-level 8

inactive-mode

iface "eth0" {

 class {
   pool 2000::/64
 }
}
\end{lstlisting}

\subsubsection{Example 16: Leasequery}
A separate entity
called requestor can send queries regarding assigned addresses and
prefixes. Server can be configured to support such lease queries.
See section \ref{feature-leasequery} for detailed explanation.

\begin{lstlisting}
#server.conf

log-level 8

iface "eth0" {
 accept-leasequery

 class {
   pool 2000::/64
 }
}
\end{lstlisting}


\subsubsection{Example 17: Authentication}
\label{example-server-auth}
It is possible to configure server to require authentication. In this
example, HMAC-SHA-512 will be used as an authentication method.
Key Generation Nonce will have 64 bytes.

\begin{lstlisting}
# server.conf

auth-method digest-hmac-sha512
auth-key-len 64

iface eth0
{
  class
  {
    pool 2000::100-2000::10f
  }
}
\end{lstlisting}

\subsubsection{Example 18: Relay support with unknown interface-id}
\label{example-server-relay3}
To get more informations about relay configuration, see section \ref{feature-relays}.
In pervious examples (\ref{example-server-relay1},
\ref{example-server-relay2}) it was assumed that interface-id set by
relay is known. However, in some cases that is not true. If sysadmin
wants to accept relayed messages from any relay, there is a feature
called guess mode. It tries to match any relay defined in server.conf
instead of exactly checking interface-id value.

Since there is only one relay defined, it will be used, regardless of
the interface-id value (or even lack of thereof).

\begin{lstlisting}
# server.conf
guess-mode

iface relay1 {
  relay eth1
  interface-id 5020
  class {
    pool 2000::1-2000::ff
  }
  option dns-server 2000::100,2000::101
}
\end{lstlisting}


\subsubsection{Example 19: DS-Lite tunnel (AFTR)}
Server is able to provide Dual-Stack lite configuration for clients.
Both address and name based configurations are supported:

\begin{lstlisting}
iface "eth0" {
 class {
   pool 2001:db8::/64
 }

 option ds-lite 2001:db8:1::ffff
 option ds-lite sc.example.org
}
\end{lstlisting}

\subsubsection{Example 20: Custom options}
\label{example-server-custom}
Server may be configured to also provide custom options to the
clients. See Section \ref{feature-custom-options} for details.

\begin{lstlisting}
iface "eth0" {
 class {
   pool 2001:db8::/64
 }
 option 145 duid 01:02:a3:b4:c5:dd:ea
 option 146 address 2001:db8:1::dead:beef
 option 147 address-list 2001:db8:1::aaaa,2001:db8:1::bbbb
 option 148 string "secretlair.example.org"
}
\end{lstlisting}


\subsubsection{Example 21: Remote Autoconfiguration}
Server does support experimental extension called remote
autoconfiguration, as defined in \cite{draft-remote-autoconf}. See
Section \ref{feature-remote-autoconf} for details and configuration
examples.
